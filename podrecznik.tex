\documentclass[]{wl}
%
        %
%
        \newif\ifflaglessadvertising \IfFileExists{morefloats.sty}{
                        \usepackage{morefloats}
                    }{} \def\authors{Piotr Czerniawski, Jarosław Lipszyc, Marcin Wilkowski}
        \author{\authors}
        \title{Otwórz się!}
        \def\translatorsline{}
%
        \def\bookurl{http://prawokultury.pl/publikacje/podrecznik}
%
        \def\rightsinfo{Ten utwór nie jest chroniony prawem autorskim i~znajduje się w~domenie
            publicznej, co oznacza że możesz go swobodnie wykorzystywać, publikować
            i~rozpowszechniać. Jeśli utwór opatrzony jest dodatkowymi materiałami
            (przypisy, motywy literackie etc.), które podlegają prawu autorskiemu, to
            te dodatkowe materiały udostępnione są na licencji
            \href{http://creativecommons.org/licenses/by-sa/3.0/}{Creative Commons
            Uznanie Autorstwa – Na Tych Samych Warunkach 3.0 PL}.}
        %
            \def\rightsinfo{Ten utwór jest udostępniony na licencji\\
            \href{http://creativecommons.org/licenses/by-sa/3.0}{Creative Commons Uznanie autorstwa - Na tych samych warunkach 3.0}.}
        %
%
        \def\sourceinfo{
            }
        \def\description{Publikację dofinansowano ze środków Ministerstwa Kultury i~Dziedzictwa Narodowego}
\begin{document}
\maketitle{} \tytul{ \autorutworu{Piotr Czerniawski} \autorutworu{Jarosław Lipszyc}
\autorutworu{Marcin Wilkowski} \nazwautworu{Otwórz się!} \translatorsline{}
} \powiesc{ \naglowekrozdzial{Wstęp} \akap{Jeśli czytasz ten tekst, to zapewne już teraz masz problem z~prawem autorskim. Zapewne masz go od niedawna. I~zapewne zastanawiasz się, dlaczego kiedyś ten problem dla ciebie nie istniał.}
\akap{Większość praw, które znamy, dotyczy pewnego tylko, szczególnego wycinka rzeczywistości. Prawo autorskie jest inne: reguluje całość relacji komunikacyjnych między ludźmi. Każdy email, rozmowa telefoniczna, czat czy wpis na stronie internetowej jest regulowany prawem autorskim. To ono mówi nam, co jest dozwolone, a~co jest zakazane. Dlatego prawo autorskie ma w~krajobrazie współczesnej kultury funkcję wyjątkową.}
\akap{Jeszcze nie tak dawno temu komunikowanie się z~innymi było stosunkowo proste: wszystko to, co prywatne — jak rozmowy, listy, telefony — było dozwolone. Wszystko to, co związane z~publikacją — a~więc należące do sfery publicznej — wymagało zarządzenia kwestią praw autorskich, ale nie musieliśmy się tym specjalnie przejmować, gdyż obowiązek ten brał na siebie wydawca.}
\akap{Pojawienie się komputerów i~internetu dramatycznie zmieniło sytuację. Dzięki nim każdy z~nas stał się publikującym autorem i~wydawcą zarazem. To oznacza także, że każdy z~nas jest osobiście odpowiedzialny za przestrzeganie prawa autorskiego.}
\akap{Niniejszy poradnik ma za zadanie wprowadzić nas w~świat podstawowych pojęć i~regulacji prawa autorskiego. Wprowadzenie to ma charakter skrótowy, i~nie rości sobie pretensji do pełnego rozpoznania problemów wynikających z~tej regulacji. Niestety, w~większości przypadków nie pozwoli nam na zrezygnowanie z~konsultacji prawnika. Będziemy jednak przynajmniej wiedzieć jak zadać pytanie i~pozwoli nam zrozumienie jego odpowiedzi.}
\naglowekrozdzial{1. Utwory} \akap{Pojęcie utworu jest jednym z~podstawowych elementów systemu prawa autorskiego. W~potocznym rozumieniu za utwór uznajemy przede wszystkim wynik jakiejś działalności o~artystycznym charakterze, taki jak film, obraz, książka czy piosenka. Nie mamy też problemów w~uznaniu za utwór artykułów naukowych, rozmaitych broszur, opracowań, analiz czy tekstów publikowanych w~prasie: chociaż nie mają one charakteru artystycznego, na pewno muszą być wynikiem jakiejś twórczej pracy. Zaproponowana w~ustawie o~prawie autorskim definicja utworu jest jednak zdecydowanie szersza.}
\naglowekpodrozdzial{Kiedy mamy do czynienia z~utworem?} \akap{Przy określaniu, czy rezultat twórczej aktywności autora jest utworem w~sensie prawnym, nie ma znaczenia wartość artystyczna lub materialna jego pracy ani też cel jej powstania. Nawet najbardziej grafomański wiersz w~systemie prawa autorskiego uzyska taką samą ochronę jak poezja Miłosza. Nie chodzi tu przecież o~artyzm, ale o~kwestie formalne. W~tekście ustawy czytamy, że utworem jest:}
\dlugicytat{ \akap{każdy przejaw działalności twórczej o~indywidualnym charakterze, ustalony w~jakiejkolwiek postaci, niezależnie od wartości, przeznaczenia i~sposobu wyrażenia.}
} \akap{Ustawa mówi, że utwór musi być \wyroznienie{ustalony}. Podpowiada to zresztą zdrowy rozsądek — w~jaki sposób można byłoby chronić utwory istniejące wyłącznie w~wyobraźni twórcy? Wiersz jest utworem tylko wtedy, gdy można go przeczytać lub wysłuchać, monodram — o~ile widzowie mogą go zobaczyć lub istnieje jego scenariusz, piosenka — o~ile może być usłyszana. Co prawda sformułowanie „forma ustalona” ma dość szerokie znaczenie, ale w~praktyce wystarczy przyjąć, że utworem jest informacja w~jakiś sposób uzewnętrzniona — jako nagranie na taśmie magnetycznej, obraz na kliszy fotograficznej, plik na komputerze, zapis na kartce papieru, ale też jako wykonanie nieprowadzące do powstania materialnego nośnika (wygłoszenie, zaśpiewanie itd.) i~tak dalej.}
\akap{Drugą z~cech pozwalających uznać, że mamy do czynienia z~utworem, jest indywidualny charakter działalności twórczej. Aby zrozumieć tę zasadę spróbujmy zanalizować przykład, w~którym ktoś narysował w~programie graficznym taki oto samochód:}
\ilustr{samochod.png}{Rysunek samochodu} \akap{Przyjmuje się, że działalność twórcza musi zawierać znamiona oryginalności i~indywidualności. To, że rysunek jest oryginalny nie oznacza, że musi być dziełem sztuki. Nasz przykład — chociaż pewnie podobny do wielu innych — nie jest kopią, nie został też w~żaden sposób sztucznie wygenerowany i~w~tym właśnie sensie jest oryginalny. Posiada też swój indywidualny charakter — styl rysowania, dobór perspektywy, kolorów i~inne jego cechy wynikają bezpośrednio z~wyborów rysownika, są odbiciem jego — mniej lub bardziej świadomego, subiektywnego i~twórczego wyboru.}
\akap{Warto podkreślić, że \wyroznienie{prawo autorskie nie chroni idei}
(w naszym przykładzie — idei samochodu), lecz jedynie \wyroznienie{sposób wyrażenia idei}
(którym w~naszym przykładzie jest właśnie rysunek samochodu). Utworem nie jest przy tym kartka papieru, fizyczne ślady pociągnięć ołówkiem układające się w~wizerunek samochodu. Kartka papieru to tylko nośnik informacji, medium przy pomocy którego rozpowszechniany jest niematerialny utwór.}
\akap{Dlatego kupując książkę nie zyskujemy prawa do swobodnego rozpowszechniania jej treści. Możemy oczywiście dysponować samym przedmiotem, na przykład sprzedać przeczytaną książkę na aukcji internetowej. Jest to wyraźnie dozwolone w~prawie autorskim, które stwierdza, że po wprowadzeniu utworu do obrotu prawo do zezwalania na dalszy obrót takimi egzemplarzami ulega wyczerpaniu. Ale sam niematerialny utwór pozostaje objęty
\wyroznienie{monopolem prawnoautorskim}. Oznacza to m.in., że nie możemy sprzedawać samodzielnie zrobionych kopii kupionej przez nas książki.}
\akap{Omawiamy koncepcję utworu na przykładzie rysunku stworzonego na komputerze, jednak w~ustawie znaleźć można wiele innych przykładów utworów. System prawa autorskiego obejmuje utwory:}
\dlugicytat{ \akap{wyrażone słowem, symbolami matematycznymi, znakami graficznymi (literackie, publicystyczne, naukowe, kartograficzne oraz programy komputerowe), plastyczne, fotograficzne, lutnicze, wzornictwa przemysłowego, architektoniczne, architektoniczno\dywiz{}urbanistyczne i~urbanistyczne, muzyczne i~słowno\dywiz{}muzyczne, sceniczne, sceniczno\dywiz{}muzyczne, choreograficzne i~pantomimiczne, audiowizualne (w tym filmowe)}
} \akap{Trudno dziś wyobrazić sobie, jakie jeszcze kształty może przyjąć w~przyszłości ludzka kreatywność, dlatego system prawa autorskiego obejmuje awansem wszystkie potencjalnie możliwe formy ustalenia utworu. Do katalogu utworów włączone zostały też zbiory artykułów czy fotografii, antologie i~bazy danych, o~ile zostały zbudowane na podstawie twórczego wyboru ich poszczególnych elementów.}
\naglowekpodrozdzial{Rodzaje utworów} \akap{Ustawa wprowadza rozróżnienie między utworami pierwotnymi i~utworami zależnymi, określanymi też czasem jako opracowania czy adaptacje. Nasz przykładowy rysunek samochodu jest utworem pierwotnym — gdybyśmy wydrukowali go na domowej drukarce lub zrobili jego kserokopię nadal nie zmieniłby się jego status — byłby to po prostu kolejny egzemplarz tego samego utworu. Utworem zależnym stałby się dopiero, gdyby — przykładowo — ktoś inny w~programie graficznym pokolorował obrazek, dodał mu nowe elementy i~wykorzystał do stworzenia plakatu albo mema (nawet jeśli miałby nie okazać się szczególnie udany):}
\ilustr{szofere.png}{Pokolorowany rysunek samochodu z napisem „Jestę szoferę”}
\akap{Utwór zależny ma w~sobie jakiś twórczy wkład nowego autora, ale korzysta też z~elementów utworu oryginalnego (macierzystego). Przykłady utworów zależnych wypisaliśmy w~tabelce:}
\par{} \vspace{1em} { \raggedright{}
\begin{tabularx}{\textwidth{}}{|X|X|X|X|X|}
\hline{} utwór pierwotny&utwór zależny\\\hline \tytuldziela{Hobbit, czyli tam i~z~powrotem}
J. R. R. Tolkiena (1937)&Tłumaczenie Marii Skibiniewskiej (1960)\\\hline
\tytuldziela{Wiedźmin} Andrzeja Sapkowskiego (1990–1999)&film \tytuldziela{Wiedźmin}
(2001, reż. Marek Brodzki)\\\hline Assassin's Creed — seria gier komputerowych&\tytuldziela{Assassin's Creed: Renesans}
— książka Olivera Bowden\\\hline
\end{tabularx}
} \vspace{1em} \akap{Utworem zależnym może być więc adaptacja, tłumaczenie, remiks czy przeróbka. Utwory zależne podlegają ochronie niezależnie od utworów pierwotnych — chronione jest więc tłumaczenie Marii Skibniewskiej, ale równolegle chronione są prawa spadkobierców Tolkiena do oryginału.}
\akap{Świadomość istnienia kategorii utworów zależnych jest bardzo ważna dla zrozumienia działania systemu prawa autorskiego. Tłumaczenie Hobbita mogło być wydane jedynie za zgodą posiadaczy praw do tłumaczenia i~jednocześnie za zgodą posiadaczy praw do oryginału. Rozpowszechnianie utworu zależnego wymaga bowiem zgody wszystkich uprawnionych.}
\naglowekpodrozdzial{Kiedy nie mamy do czynienia z~utworem?} \akap{Zadaniem prawa autorskiego jest określenie zasad rozpowszechniania i~korzystania z~utworów, dlatego jesteśmy nieustannie w~zasięgu jego oddziaływania: kiedy czytamy gazetę, oglądamy telewizję, słuchamy muzyki czy komentujemy w~portalach społecznościowych. Na szczęście istnieją wyjątki od tej wszechobecności utworów — wskazują je konkretne przepisy ustawy o~prawie autorskim.}
\akap{Utworami nie są więc akty normatywne (np. ustawy) lub ich projekty, dokumenty urzędowe, urzędowe materiały, znaki i~symbole, opublikowane opisy patentowe lub ochronne oraz proste informacje prasowe, niezależnie od tego jak duży byłby wkład indywidualnej twórczości w~nich zawarty. Prawo autorskie nie chroni ponadto idei, odkryć, procedur, metod i~zasad działania oraz koncepcji matematycznych. Gdyby było inaczej, każdorazowe wykorzystanie w~obliczeniach dostępnego dziś powszechnie wzoru wymagałoby zezwolenia jego twórców i~uiszczenia odpowiedniej opłaty, co nie miałoby sensu.}
\naglowekpodrozdzial{Digitalizacja a~utwór} \akap{Instytucje takie jak muzea, biblioteki czy archiwa prowadzą obecnie wiele programów digitalizacyjnych: skanują książki, dokumenty i~fotografie — część z~nich publikowana jest online. Wokół tych działań raz po raz stawiane są pytania dotyczące tego, czy w~akcie skanowania powstaje utwór zależny.}
\akap{Jeśli skan wykonany został w~sposób automatyczny bez żadnego twórczego wkładu ze strony bibliotekarza, nie jest utworem — podobnie utwór nie powstaje w~trakcie kserowania czy wydruku. Jeśli jednak w~trakcie digitalizacji opracowano zdjęcie, usunięto jego niedoskonałości, poprawiono kolory czy w~inny twórczy sposób przetworzono fotografię, możemy mieć do czynienia z~utworem objętym standardową ochroną.}
\akap{Na podobnej zasadzie fotografując obraz w~muzeum nie tworzymy nowego utworu — o~ile nie jest to fotografia artystyczna, przygotowana za pomocą odpowiednio kreatywnych ujęć czy filtrów.}
\naglowekpodrozdzial{FAQ} \listanum{ \punkt{ \akap{ \wyroznienie{Czy grafika istniejąca wyłącznie w~formie cyfrowej może być utworem?}
} \akap{Nie ma znaczenia, czy utwór powstaje jedynie w~formie cyfrowej (\slowoobce{born digital}) czy też analogowej (wydruk). Każdy, kto podejdzie do ekranu komputera może zapoznać się ze stworzoną w~programie graficznym pracą — wypełniona jest tutaj konieczność ustalenia utworu.}
} \punkt{ \akap{ \wyroznienie{Czy programy komputerowe są utworami?}
} \akap{Tak, programy komputerowe są utworami i~podlegają ochronie tak jak utwory literackie czy wizualne. Ochronie prawa autorskiego podlegają także bazy danych, o~ile można doszukać się wkładu twórczego na poziomie samej bazy, a~nie tylko poszczególnych jej elementów. Nietwórcze bazy danych mogą natomiast podlegać ochronie na podstawie osobnej ustawy — o~ochronie baz danych.}
} \punkt{ \akap{ \wyroznienie{Czy jeśli fotografia dostępna w~internecie nie ma oznaczenia \textcopyright{} nie podlega ochronie prawnoautorskiej?}
} \akap{Utwory podlegają ochronie wynikającej z~przepisów prawa autorskiego bez względu na to, czy są w~ten sposób oznaczone czy nie. Nie muszą być także nigdzie rejestrowane, dlatego utworami w~internecie mogą być także komentarze na forach, nagrania na automatyczną sekretarkę itd.}
} } \naglowekrozdzial{2. Kto jest twórcą?} \akap{Ktoś mógłby pewnie uznać prawo autorskie za niedostępną przestrzeń pełną abstrakcyjnych pojęć i~nieżyciowych zasad, które dla zwykłego człowieka nie mają większego znaczenia, ponieważ dotyczą wyłącznie artystów przez duże „A” i~twórców przez duże „T”. Problem w~tym, że — jak pokazaliśmy też w~poprzednim rozdziale — prawo to dotyczy i~oddziałuje na wszystkich nas nieustannie. Ciągle jesteśmy otoczeni utworami i~sami — czasami całkowicie nieświadomie — stajemy się twórcami.}
\akap{Jeśli jeszcze raz zajrzymy do definicji utworu zapisanej w~pierwszych paragrafach przepisów prawa autorskiego uświadomimy sobie, że dziś tak naprawdę bardzo trudno nie być twórcą. Aby uzyskać ten status nie trzeba wystawiać swoich dzieł w~galeriach ani zbierać pochlebnych recenzji w~czasopismach literackich. Czasem wystarczy po prostu zrobić zdjęcie komórką albo dodać kilkuzdaniowy komentarz na forum. To, czy stajemy się twórcą nie zależy ani od naszej woli, ani od naszego wieku:}
\dlugicytat{ \akap{Zosia ma trzy lata i~kredkami próbowała narysować portret mamy. Na kartce pojawiła się duże koło z~dwoma punktami w~różnych miejscach (głowa i~oczy), jakaś kreska udająca usta oraz masa chaotycznych linii. Zosia jest twórcą, chociaż nawet nie zna jeszcze liter.}
} \akap{Oryginalny, posiadający indywidualny charakter rysunek Zosi, będący odbiciem jej dziecięcej kreatywności, podlega ochronie prawnoautorskiej tak, jak rysunki wielkich mistrzów malarstwa. W~systemie prawa autorskiego utwór może powstać bez względu na wiek, poczytalność, umiejętności czy motywacje twórcy. Formalnie nie możemy z~tej ochrony zrezygnować, a~przecież nie zawsze wrzucając zdjęcie na bloga czy komentarz na forum zależy nam, żeby podlegały one jakiemuś szczególnemu zabezpieczeniu — pewnie nawet zazwyczaj wcale o~tym nie myślimy.}
\akap{Warto jednak dodać, że twórcą może być jedynie osoba fizyczna. Nie może być twórcą instytucja, państwo, czy przedsiębiorstwo:}
\dlugicytat{ \akap{Broszura wydana przez lokalny samorząd (np. urząd miasta) nie jest utworem samorządu, ale osób, które nad nią pracowały.}
} \akap{Dobrze jest o~tym wspomnieć, bo pozwala to uchwycić bardzo ważną sprawę: różnicę między byciem twórcą (autorem), a~byciem uprawnionym z~tytułu majątkowych praw autorskich do utworu. Jeśli redaktorzy tworzący broszurę na zlecenie samorządu podpiszą z~nim umowę o~przeniesienie praw majątkowych, to samorząd zostanie uprawnionym, ale autorzy broszury jednak wciąż pozostaną twórcami.}
\akap{W przepisach prawa autorskiego znajdują się odniesienia do wielu osób mających różny związek z~utworem. Ich status i~idące za tym prawa będą różne, np.:}
\par{} \vspace{1em} { \raggedright{}
\begin{tabularx}{\textwidth{}}{|X|X|X|X|X|}
\hline{} autor (twórca)&Utwór to przejaw jego twórczej działalności o~indywidualnym charakterze. Posiada pełne prawo do dysponowania utworem, o~ile nie przekazał części swoich praw w~ramach odpowiedniej umowy albo w~ramach spadku.\\\hline
współautor (współtwórca)&Utwór może być wynikiem działalności kilku osób, jeżeli działając w~porozumieniu wnoszą swoje twórcze i~indywidualne wkłady tworząc tym samym jedno dzieło. Wtedy prawa przysługują takim osobom wspólnie.\\\hline
uprawniony inny niż twórca&Osoba, na której rzecz twórca przekazał te prawa, np. w~drodze umowy lub dziedziczenia. W~niektórych przypadkach prawa autorskie powstają od razu na rzecz osoby innej niż twórca (np. programy komputerowe tworzone przez pracowników w~ramach ich obowiązków — uprawnionym z~mocy ustawy staje się pracodawca).\\\hline
licencjobiorca&osoba, której uprawniony zezwolił na korzystanie z~utworu bez przenoszenia na nią praw autorskich.\\\hline
każdy&Posiada prawo do korzystania z~utworu w~ramach dozwolonego użytku. Więcej na ten temat w~rozdziale 9~i~10.\\\hline
twórca utworu zależnego&Posiada prawo do stworzenia i~rozpowszechniania utworu zależnego na mocy umowy z~twórcą utworu oryginalnego. Więcej informacji o~utworach zależnych znaleźć można w~rozdziale 11.\\\hline
inne podmioty praw pokrewnych&Artysta wykonawca, producent fonogramu/wideogramu, nadawca, prawa do pierwszych wydań, wydań naukowych i~krytycznych\\\hline
\end{tabularx}
} \vspace{1em} \akap{Twórcy i~inni uprawnieni posiadają dość szczególne przywileje. Opisujemy je w~kolejnych dwóch rozdziałach.}
\naglowekpodrozdzial{FAQ} \listanum{ \punkt{ \akap{ \wyroznienie{Kiedy twórca nie jest podmiotem praw majątkowych?}
} \akap{Zasadą jest, że prawa majątkowe powstają pierwotnie na rzecz twórcy. Twórca może je następnie przenieść na inne osoby — musi to jednak zrobić za pomocą odpowiedniej pisemnej umowy. Szczególnym przypadkiem jest stworzenie utworu w~ramach wykonywania umowy o~pracę, gdzie prawo przewiduje łatwiejsze nabywanie praw przez pracodawcę. Jeżeli natomiast prawa pozostaną przy twórcy, to jego spadkobiercy mogą je nabyć w~drodze dziedziczenia. Natomiast udzielenie przez twórcę licencji innej osobie nie prowadzi do wyzbycia się przez niego praw majątkowych. O~licencjach piszemy w~kolejnych rozdziałach.}
\akap{Istnieją też określone w~ustawie przypadki, gdy twórca od samego początku nie jest podmiotem praw majątkowych do swojego utworu. Takim wyjątkiem są programy komputerowe stworzone przez pracowników w~ramach wykonywania umów o~pracę. Całość praw do takiego programu przysługuje pracodawcy od chwili ich powstania. Natomiast prawa osobiste zawsze przysługują twórcy i~nie mogą być przenoszone. Co ciekawe, istnieją one także po śmierci twórcy. O~prawach osobistych, majątkowych oraz o~umowach piszemy w~rozdziałach 4, 5, 6~i~14.}
} \punkt{ \akap{ \wyroznienie{Czy przysługują mi prawa autorskie do mojej pracy magisterskiej?}
} \akap{Prawo autorskie przysługuje tu w~pełni autorowi pracy, chociaż uczelnia ma pierwszeństwo w~opublikowaniu jej.}
} \punkt{ \akap{ \wyroznienie{Jestem pracownikiem, do którego obowiązków należy fotografowanie. Czy przysługują mi prawa majątkowe do wykonanych przeze mnie zdjęć?}
} \akap{Jeśli fotografie zostały wykonane w~ramach obowiązków zapisanych w~umowie między pracownikiem a~pracodawcą (umowy o~pracę) to autorskie prawa majątkowe przysługują pracodawcy w~granicach opisanych w~umowie. Zatrudnienie na innej podstawie (np. umowa o~dzieło) powoduje przejście praw na zleceniodawcę tylko wtedy, gdy umowa jest zawarta w~formie pisemnej i~zawiera wyraźne postanowienie o~przeniesieniu praw wraz ze wskazaniem pól eksploatacji.}
} } \naglowekrozdzial{3. Prawa twórców} \akap{Prawo autorskie reguluje wszelkie sposoby korzystania z~utworów objętych monopolem prawnoautorskim. Regulacja ta dotyczy zarówno kwestii związanych z~samym autorstwem utworu, jak i~związanych z~wykonywaniem monopolu prawnoautorskiego. Dlatego prawa autorskie dzielimy na dwa podstawowe rodzaje: prawa osobiste i~prawa majątkowe. Zobaczmy czym się różnią:}
\dlugicytat{ \akap{Ania jest autorką powieści. Swój rękopis wydaje w~wydawnictwie, które wykupiło od niej prawa do tekstu i~może nimi swobodnie rozporządzać.}
} \akap{W systemie prawa autorskiego Ania na zawsze uzyskuje status autorki powieści. Nikt nie jest w~stanie odebrać jej tego statusu. Nawet jeśli po podpisaniu umowy z~wydawnictwem nie może już sama rozporządzać swoim tekstem, bo uzyskało ono monopol na jego wykorzystanie, to wydawnictwo nadal ma obowiązek oznaczać go imieniem i~nazwiskiem autorki oraz zachować jego oryginalną treść. Dzięki podpisaniu umowy o~przeniesieniu praw autorskim wydawnictwo może dysponować powieścią Ani przez cały czas jej życia i~przez 70 lat od daty jej śmierci. Po tym okresie wydawnictwo straci monopol na rozpowszechnianie tego utworu i~zarabianie na nim — odtąd wszyscy mogą korzystać z~niego bez przeszkód. Różnice między osobistymi a~osobistymi prawami autorskimi pokazuje poniższa tabela;}
\par{} \vspace{1em} { \raggedright{}
\begin{tabularx}{\textwidth{}}{|X|X|X|X|X|}
\hline{} Autorskie prawa osobiste&Autorskie prawa majątkowe\\\hline
bezpośrednio związane z~osobą autor&pośrednio związane z~osobą autor\\\hline
dotyczą relacji twórcy z~utworem&dotyczą dysponowania utworem i~zarabiania na jego rozpowszechnianiu lub innej formie wykorzystani\\\hline
umożliwiają rozpoznanie utworu&dają prawo kontroli nad korzystaniem z~utworu\\\hline
nieograniczone w~czasie&ograniczone w~czasie (standardowo trwają przez 70 lat od śmierci autora lub od daty rozpowszechnienia utworu)\\\hline
nie można z~nich zrezygnować ani nikomu przekazać, zawsze przysługują twórc&mogą być przeniesione przez autora w~ramach umowy o~przeniesieniu praw lub w~ramach spadku, we wskazanych w~ustawie przypadkach przysługują innej osobie niż twórca, uprawniony z~autorskich praw majątkowych może upoważnić inną osobę do korzystania z~nich (licencja)\\\hline
\end{tabularx}
} \vspace{1em} \akap{Charakterystykę praw osobistych i~majątkowych przedstawiamy w~kolejnych rozdziałach.}
\naglowekrozdzial{4. Autorskie prawa osobiste} \akap{Prawa osobiste to bardzo ważna kategoria praw autorskich. Określają one podstawowe zasady wykorzystywania utworów, a~przy tym są to prawa wieczne i~niezbywalne. Spróbujmy opisać niektóre z~nich.}
\akap{Każdy twórca ma prawo do \wyroznienie{autorstwa}, czyli do uznania przez każdego jego relacji z~utworem wynikającej z~faktu jego stworzenia. Z~prawem tym powiązane jest kolejne autorskie prawo osobiste —
\wyroznienie{prawo oznaczenia utworu swoim nazwiskiem lub pseudonimem albo do udostępniania go anonimowo}. Bolesław Prus napisał
\tytuldziela{Lalkę} ponad 100 lat temu, ale nadal każdy ma obowiązek podpisania tego utworu jego imieniem i~nazwiskiem (a właściwie to przyjętym przez niego już na początku pracy dziennikarskiej pseudonimem). Szczególnym przypadkiem naruszeniem prawa do autorstwa jest plagiat: przypisanie sobie autorstwa utworu stworzonego przez kogoś innego. Ale nie tylko o~plagiat tu chodzi. Także usunięcie informacji o~autorze jest naruszeniem prawa do autorstwa, nawet jeśli nie popełniamy przy tym plagiatu.}
\akap{Autor ma prawo do autorstwa swojego utworu, a~nie jakiegoś innego. Jeśli tworzymy utwór zależny, to musimy zawsze prawidłowo oznaczyć co pochodzi z~utworu oryginalnego, a~co jest naszym wkładem.}
\akap{Po drugie twórca posiada wyłączne prawo do podjęcia \wyroznienie{decyzji o~pierwszym udostępnieniu swojego dzieła publiczności}. Tylko autor decyduje, czy przechowywane w~szufladzie albo na komputerze materiały zostaną pokazane komuś jeszcze, czy rękopis powieści powędruje do wydawnictwa, a~nagranie z~komórki trafi do internetu. Jeśli ma w~głowie pomysł na scenariusz czy melodię piosenki, tylko od niego zależy, czy podzieli się nią z~kimś innym. Nie wolno nam rozpowszechniać utworu wbrew woli jego twórcy.}
\akap{Autor ma też prawo do \wyroznienie{integralności} swoich utworów, nawet jeśli prawa majątkowe do nich zostały przekazane komuś innemu lub wygasły. W~przypadku utworów z~domeny publicznej tego prawa nie można egzekwować, jeśli nie ma już z~kim konsultować wprowadzanych zmian (nie ma dostępu do spadkobierców autora).}
\akap{Ochrona praw osobistych nie ma na celu zabezpieczenia majątkowego interesu twórcy, choć korzystanie z~nich lub naruszenie może wiązać się z~takimi konsekwencjami (np. wypłata zadośćuczynienia za krzywdę wywołaną zniekształceniem utworu, zapłata wyższego wynagrodzenia z~tytułu sporządzenia projektu budynku przez uznanego, a~nie początkującego architekta itd.). Celem tej ochrony jest poszanowanie godności osobistej twórcy wyrażającej się w~jego intelektualnym związku z~dziełem. Chociaż brzmi to dość patetycznie, taka ochrona może mieć zdecydowanie praktyczny wymiar — widać to szczególnie w~przypadku plagiatów, tak częstych dziś w~szkołach i~na uczelniach.}
\naglowekpodrozdzial{FAQ} \listanum{ \punkt{ \akap{ \wyroznienie{Czy jako autor mam prawo do zachowania anonimowości?}
} \akap{Przepisy prawa autorskiego dają taką możliwość. Decyzja o~upublicznieniu utworu bez informacji o~autorze może generować jednak wiele problemów związanych z~jego rozpowszechnianiem i~wykorzystaniem.}
} \punkt{ \akap{ \wyroznienie{Czy mogę sprzedać prawo do bycia oznaczonym jako autor utworu?}
} \akap{W polskim prawie taki krok nie jest możliwy — prawa osobiste są niezbywalne. Nawet gdy twórca nie ma nic przeciwko podpisaniu jego utworu przez kogoś innego, to będzie to w~świetle polskiego prawa nielegalne. Dlatego w~Polsce nie można wykonywać legalnie zawodu ghost writera, choć w~praktyce ograniczenie to jest powszechnie ignorowane.}
} \punkt{ \akap{ \wyroznienie{Czy mogę zgodnie z~prawem zlecić napisanie pracy naukowej, którą przedstawię jako własną?}
} \akap{Nie. Autorskie prawa osobiste wymagają podania prawidłowych informacji o~autorze, i~od tej zasady nie ma wyjątków. Nawet jeśli twórca chce pozostać anonimowy, to my nadal nie mamy prawa do podpisania się pod jego utworem. Przedstawienie się jako autor utworu anonimowego również jest naruszeniem zasady rozpoznania autorstwa.}
} } \naglowekrozdzial{5. Autorskie prawa majątkowe} \akap{Istotą majątkowych praw autorskich jest przyznanie ich posiadaczowi monopolu na korzystanie z~utworów. Monopol ten ma jednak ograniczony zakres. Na przykład, dzięki przepisom o~dozwolonym użytku każdy z~nas ma prawo w~pewnym zakresie korzystać z~utworu bez konieczności uzyskania zgody posiadacza praw majątkowych, a~pewne wykorzystania zgody takiej wymagają. Czym jest korzystanie z~utworu?}
\dlugicytat{ \akap{Przeczytanie książki, obejrzenie filmu, wysłuchanie koncertu, odtworzenie plyty itp.}
} \akap{jest oczywiście korzystaniem z~utworu, ale nie podlega kontroli i~ograniczeniom ze strony posiadaczy praw majątkowych. Problemy zaczynają się tam, gdzie mamy do czynienia ze zwielokrotnianiem, modyfikacją lub rozpowszechnianiem utworu, czyli w~sytuacjach takich jak:}
\dlugicytat{ \akap{kserowanie książki, cytowanie, nagrywanie audycji radiowych lub programów telewizyjnych, udostępnienie w~internecie filmów czy zdjęć, publikacja tłumaczenia artykułu z~prasy zagranicznej czy odtworzenie piosenki w~radiu}
} \akap{Część z~tych działań wymaga uzyskania zgody właściciela praw autorskich i~wypłaty na jego rzecz odpowiedniego wynagrodzenia, a~inne są zawsze legalne. Szczegółowo wyjaśnimy to w~kolejnym rozdziale.}
\akap{Autorskich prawa majątkowe są, w~przeciwieństwie do praw osobistych, ograniczone w~czasie. Po upływie określonego czasu utwory przechodzą na własność całego społeczeństwa — stają się elementem domeny publicznej (więcej na ten temat w~rozdziale 7). Standardowo ochrona autorskich praw majątkowych — czyli faktyczny monopol na kontrolę nad utworem — trwa przez 70 lat od daty śmierci twórcy. Jeśli utwór jest anonimowy, okres ochrony liczy się od daty pierwszego rozpowszechnienia. W~przypadku utworów audiowizualnych (filmów) okres obowiązywania ochrony liczy się od daty śmierci najpóźniej zmarłej z~wymienionych osób: głównego reżysera, autora scenariusza, autora dialogów, kompozytora muzyki skomponowanej na potrzeby takiego utworu.}
\naglowekpodrozdzial{Okres ochrony utworu} \akap{W całej Unii Europejskiej, w~tym także w~Polsce, jest to 70 lat od śmierci twórcy lub daty upublicznienia utworu.}
\dlugicytat{ \akap{Okres po śmierci autora liczony jest od ostatniego dnia roku kalendarzowego. Dlatego dzieła Tadeusza Boya\dywiz{}Żeleńskiego, który zmarł 4~lipca 1941 roku, do domeny publicznej weszły dopiero 1~stycznia 2012 roku. Na całym świecie 1~stycznia obchodzony jest jako Dzień Domeny Publicznej. Więcej informacji na ten temat znaleźć można w~rozdziale 7).}
} \akap{W przypadku utworów, których twórcami jest wiele osób (np. autorzy słów i~muzyki do piosenki), utwór objęty jest monopolem przez 70 lat od śmierci najpóźniej zmarłego twórcy.}
\akap{Trzeba pamiętać, że oprócz praw majątkowych twórcy istnieją także prawa pokrewne: prawa majątkowe producentów fonogramów/wideogramów, prawa artystów wykonawców czy prawa majątkowe związane z~pierwszymi czy krytycznymi wydaniami utworów z~domeny publicznej. Obecnie czas ich trwania to 25 do 50 lat od pierwszego rozpowszechnienia.}
\naglowekpodrozdzial{Wydłużanie czasu ochrony} \akap{Pierwszy akt prawny dotyczący prawa autorskiego uchwalony został w~1710 roku przez parlament Wielkiej Brytanii. Był to „\tytuldziela{Statut Anny}” (od imienia królowej Anny Stuart). Sformułowano w~nim bardzo ważną zasadę mówiącą, że ochrona prawna utworu powinna obowiązywać tylko przez pewien ograniczony czas. Zapisy statutu mówiły o~14 latach od pierwszej publikacji.}
\akap{W XIX wieku okres ochrony dzieła prawami autorskimi zaczął ulegać kolejnym przedłużeniom. W~międzynarodowej Konwencji Berneńskiej o~ochronie dzieł literackich i~artystycznych, którą Polska ratyfikowała w~1928 roku, przyjęto że okres ten musi trwać co najmniej 50 lat po śmierci twórcy. Identyczny okres ochrony pojawia się w~międzynarodowym Porozumieniu w~sprawie Handlowych Aspektów Praw Własności Intelektualnej (\slowoobce{Agreement on Trade\dywiz{}Related Aspects of Intellectual Property Rights}, TRIPS) z~1994 roku. Obecnie z~reguły nie spotyka się okresu ochrony trwającego krócej niż czas życia autora i~25 lat po jego śmierci.}
\akap{Z biegiem lat presja na przedłużanie monopolu prawnoautorskiego ze strony przemysłu rozrywkowego była coraz większa. Amerykańska regulacja Copyright Term Extension Act z~1998 roku znana jest powszechnie jako „prawo Myszki Miki” — została bowiem uchwalona na skutek bezpośredniej presji korporacji Walt Disney Company na rok przed przejściem postaci Myszki Miki do domeny publicznej. Najdłuższy czas trwania praw wyłącznych posiada Meksyk — od 2003 roku obowiązuje tam ustawa ustalająca okres monopolu na 100 lat po śmierci twórcy. Oznacza to, że powstające obecnie utwory mogą przejść do domeny publicznej w~Meksyku nawet po ponad 150 latach od chwili powstania.}
\naglowekpodrozdzial{FAQ} \listanum{ \punkt{ \akap{ \wyroznienie{Jestem muzykiem i~nie chcę, żeby moje nagrania były chronione przez tak restrykcyjne zasady, bo i~tak wszyscy wymieniają się moimi piosenkami w~Internecie. Co mogę zrobić?}
} \akap{Formą obejścia rygorystycznych zasad ochrony utworu, trwających faktycznie aż do 70 lat od śmierci twórcy, jest stosowanie wolnych licencji. O~licencjach piszemy w~rozdziale 15 i~rozdziałach następnych.}
} \punkt{ \akap{ \wyroznienie{Czy autorskie prawa majątkowe podlegają dziedziczeniu?}
} \akap{Autorskie prawa majątkowe podlegają dziedziczeniu. Trwają one przecież nie tylko przez całe życie autora (twórcy), ale także przez 70 lat od jego śmierci — w~tym czasie korzystają z~nich spadkobiercy właściciela autorskich praw majątkowych. Warto zwrócić uwagę, że nie muszą to być spadkobiercy autora, bo nie każdy autor jest posiadaczem tych praw.}
} } \naglowekrozdzial{6. Własność przedmiotu a~prawa autorskie}
\akap{Kupując książkę albo płytę często mówimy, że należy ona do nas. Co to jednak tak naprawdę znaczy? Odróżnienie
\wyroznienie{nośnika informacji} od \wyroznienie{informacji} jako takiej jest kluczowe dla zrozumienia sposobu działania systemu praw autorskich. Przedmiot — książka, kaseta, płyta — w~naszym przypadku jest tylko nośnikiem informacji. Jednak prawa wyłączne do informacji zazwyczaj posiada ktoś inny. Nasze prawo własności do nośnika jest ograniczone monopolem prawnoautorskim.}
\akap{W świecie analogowym najczęściej utożsamiamy informację z~jej nośnikiem. Mówiąc „książka” mamy na myśli i~fizyczny przedmiot stojący na półce, i~jej treść. Jednak z~punktu widzenia prawa są to dwie różne rzeczy. Ta sama treść może być zapisana na wiele różnych sposobów (np. na twardym dysku komputera). Prawa wyłączne dotyczą treści. Prawo własności dotyczy przedmiotu.}
\akap{Większość rzeczy, które moglibyśmy chcieć zrobić z~należącą do nas książką\dywiz{}przedmiotem, jest objęta dozwolonym użytkiem, czyli po prostu legalna. Książkę\dywiz{}przedmiot możemy przeczytać, potem pożyczyć znajomemu, a~kiedy już nam się znudzi sprzedajemy ją na aukcji internetowej. Przedmiot należy do nas, więc wolno nam nim rozporządzać. Jednak nie znaczy to, że to samo wolno nam zrobić z~książką\dywiz{}treścią. Dlatego na przykład nie możemy książki\dywiz{}treści zeskanować i~udostępnić w~internecie.}
\akap{Widać różnicę? Jeśli jeszcze nie, spróbujmy zanalizować sytuację odwrotną: co działoby się, gdyby kupno egzemplarza książki pozwalało nam na dysponowanie jej treścią?}
\dlugicytat{ \akap{Producent filmowy kupując egzemplarz jednej z~książek J. K. Rowling o~Harrym Poterze mógłby na jej podstawie nakręcić film bez pytania się autorki o~zdanie i~nie płacąc jej ani grosza (bo zapłacił już przecież za książkę).}
} \akap{To, że kupujemy książkę czy idziemy na film do kina nie oznacza, że nabywamy prawa do tych utworów. Kupując reprodukcję obrazu Picassa — a~nawet kupując oryginał Picassa! — stajemy się jedynie właścicielami jakiegoś przedmiotu. Jeśli zatem chcemy rozpowszechniać potem jego kopie musimy nabyć osobno autorskie prawa majątkowe.}
\naglowekrozdzial{7. Domena publiczna} \akap{Pojęcie domeny publicznej (ang.
\slowoobce{public domain}) zasygnalizowaliśmy już w~rozdziale 6~poświęconym autorskim prawom majątkowym. Zasługuje ona jednak na poświęcenie jej osobnego rozdziału.}
\akap{Domena publiczna to zasób utworów, które nie są — z~różnych względów — objęte autorskimi prawami majątkowymi. W~domenie publicznej są dzieła, które nigdy nie były objęte autorskim prawem majątkowym oraz te, do których ograniczenia wynikające z~tego prawa wygasły:}
\par{} \vspace{1em} { \raggedright{}
\begin{tabularx}{\textwidth{}}{|X|X|X|X|X|}
\hline{} utwory w~domenie publicznej nigdy nie objęte autorskim prawem majątkowym&utwory w~domenie publicznej, co do których autorskie prawa majątkowe już wygasł\\\hline
Odprawa posłów greckich Jana Kochanowskiego&Obrachunki fredrowskie Tadeusza Boy\dywiz{}Żeleńskiego (zm. 1941)\\\hline
Konstytucja Rzeczypospolitej Polskiej (1997)&Tekst kolędy Bóg się rodzi Franciszka Karpińskiego\\\hline
Dama z~gronostajem Leonarda da Vinci (XVI w.)&Bitwa pod Grunwaldem Jana Matejki (zm. w~1893)\\\hline
\end{tabularx}
} \vspace{1em} \naglowekpodrozdzial{Idea domeny publicznej} \akap{Początków koncepcji domeny publicznej można dopatrywać się w~prawie rzymskim, w~którym występowała szeroka kategoria
\wyroznienie{rzeczy, których nie można posiadać na własność}, udostępnionych wszystkim obywatelom. W~znaczeniu zbliżonym do dzisiejszego pojawiła się już w~angielskim Statucie Anny z~1710 roku. Według tego aktu prawnego po 14 latach od publikacji twórcy tracili prawo do dysponowania dziełem i~przechodziło ono na własność publiczną.}
\akap{Istnienie zasobu twórczości dostępnej dla całej ludzkości bez ograniczeń uważa się za jeden z~fundamentów rozwoju kultury. W~„Manifeście Domeny Publicznej”, opublikowanym w~2010 roku przez sieć Communia czytamy, że:}
\dlugicytat{ \akap{domena publiczna leży u~podstaw naszej świadomości, wyrażonej przez wspólnotę wiedzy i~kultury. Jest surowcem, dzięki któremu tworzymy nową wiedzę i~dzieła kultury (…) jest niezbędna dla zapewnienia społecznego i~ekonomicznego dobrobytu naszych społeczeństw.}
} \akap{Domenę publiczną opisywać można metaforą powszechnego dobra wspólnego, dziedzictwa kulturowego, należących do wszystkich zasobów niematerialnych o~podstawowym znaczeniu dla społeczeństwa, tak jak do wszystkich należy język, przestrzeń publiczna czy środowisko naturalne. W~artykule opublikowanym w~zbiorze
\tytuldziela{Rethinking Copyright: History, Theory, Language} (2006) prawniczka Pamela Samuelson nadaje domenie publicznej osiem wartości:}
\par{} \vspace{1em} { \raggedright{}
\begin{tabularx}{\textwidth{}}{|X|X|X|X|X|}
\hline{} Osiem wartości domeny publicznej (P. Samuelson)\\\hline
elementy niezbędne dla tworzenia nowej wiedzy (dane, fakty, pomysły, teorie)\\\hline
dostęp do zasobów dziedzictwa kulturowego, takich jak piśmiennictwo antycznej Grecji czy dzieła Mozarta\\\hline
promowanie edukacji poprzez wymianę informacji, idei i~faktów naukowych\\\hline
ciągłe innowacje dzięki wygaśnięciu patentów i~ochrony praw autorskich\\\hline
obniżenie kosztów dostępu do informacji bez konieczności ustalania właściciela i~płacenia tantiem dzięki wygaśnięciu patentów i~ochrony praw autorskich\\\hline
promowanie zdrowia i~bezpieczeństwa publicznego dzięki informacjom i~dokonaniom naukowym\\\hline
promowanie demokratycznych procesów i~wartości dzięki informacjom, przepisom ustawowym i~wykonawczym oraz opiniom prawniczym\\\hline
umożliwianie twórczego i~konkurencyjnego naśladownictwa, dzięki wygaśnięciu patentów i~ochrony praw autorskich oraz publicznemu ujawnianiu technologii, które nie kwalifikują się do ochrony patentowej\\\hline
\end{tabularx}
} \vspace{1em} \naglowekpodrozdzial{Praktyka domeny publicznej}
\akap{Zapisany w~idei domeny publicznej brak ograniczeń wynikających z~autorskich praw majątkowych oznacza całkowitą dowolność w~ich rozpowszechnianiu i~adaptacji, na przykład swobodne tworzenie i~rozpowszechnianie tłumaczeń czy remiksów. Dzieła będące w~domenie publicznej można rozpowszechniać także w~celach zarobkowych, przy czym zgodnie z~polskim prawem od 5~do 8~proc. wpływów brutto ze sprzedaży egzemplarzy takich utworów musi być przekazywane na Fundusz Promocji Twórczości.}
\akap{W polskim prawie pojęcie domeny publicznej nie występuje dosłownie. W~polskim tłumaczeniu „Konwencji berneńskiej” termin
\slowoobce{public domain} przetłumaczony został błędnie jako „publiczna własność państwa”.}
\akap{Utwory zależne zbudowane w~oparciu o~oryginały pozostające w~domenie publicznej objęte są standardową ochroną prawnoautorską. Dobrym przykładem jest tutaj słynny obraz Marcela Duchampa „L.H.O.O.Q.” (znany jako „Mona Lisa z~wąsami”) — o~ile Mona Lisa jest bezsprzecznie w~domenie publicznej, przeróbka Duchampa wciąż objęta jest autorskim prawem majątkowym, którego ochrona trwać będzie aż do 2039 roku. Na tej samej zasadzie objęte autorskim prawem majątkowym będą wariacje muzyczne oparte o~utwory Fryderyka Chopina czy obrazy przedstawiające zabytki Krakowa.}
\akap{Mimo tego, że pojęcie domeny publicznej ani razu nie zostało przytoczone w~treści ustawy o~prawie autorskim, idea dostępnych dla wszystkich, wspólnych zasobów kultury zaczyna zakorzeniać się w~świadomości społecznej. Dużą rolę pełni w~tym Internet, który dla wielu osób jest bezpośrednim doświadczeniem potencjału domeny publicznej — dzięki internetowi możemy swobodnie kopiować i~wykorzystywać dzieła klasyków kultury, do niedawna pilnie strzeżone w~murach muzeów i~bibliotek. To, że pojęcie domeny publicznej nie pojawia się wprost w~polskim prawie nie oznacza, że ona nie istnieje. Popularyzacją idei domeny publicznej zajmują się m.in. organizacje pozarządowe skupione w~Koalicji Otwartej Edukacji (KOED).}
\akap{Każdego roku 1~stycznia obchodzimy Dzień Domeny Publicznej — termin jest nieprzypadkowy, bowiem prawa autorskie wygasają wraz z~końcem roku kalendarzowego. Pierwsze w~Polsce obchody Dnia Domeny Publicznej zorganizowała Fundacja Nowoczesna Polska w~2007 roku. Podczas obchodów w~2012 roku świętowano przejście do domeny publicznej dzieł m.in. Tadeusza Boya\dywiz{}Żeleńskiego, Jamesa Joyce'a, Ignacego Paderewskiego i~Virginii Woolf.}
\naglowekpodrozdzial{FAQ} \listanum{ \punkt{ \akap{ \wyroznienie{Czy w~polskim systemie praw autorskiego istnieje jakikolwiek sposób na przeniesienie utworu do domeny publicznej przed upływem okresu ochrony?}
} \akap{Nie, nie ma takiej możliwości. Jedyną możliwością na udzielenie zgody na dowolne wykorzystanie dzieła jest udzielenie wolnej licencji.}
} \punkt{ \akap{ \wyroznienie{Jak w~prosty sposób obliczyć, kiedy dany utwór przejdzie do domeny publicznej (kiedy wygasną do niego autorskie prawa majątkowe)?}
} \akap{Dobrym narzędziem ułatwiającym sprawdzenie statusu interesującego nas utworu jest Kalkulator Domeny Publicznej, dostępny na stronie
\href{domena.koed.org.pl}{\url{domena.koed.org.pl}}. Do skorzystania z~kalkulatora konieczna jest wiedza o~dacie śmierci autora lub dacie rozpowszechnienia utworu. Ponieważ jednak okres 70 lat liczy się od różnych dat, w~zależności od różnych okoliczności, w~konkretnych przypadkach może istnieć konieczność dysponowania szerszą wiedzą o~wszystkich tych okolicznościach.}
} } \naglowekrozdzial{8. Utwory osierocone} \akap{Kiedy omawialiśmy autorskie prawa majątkowe zwracaliśmy uwagę na to, że nakładają one określone obowiązki na osoby czy instytucje, które chcą skorzystać z~chronionych nimi utworów. Monopol przyznany przez państwo podmiotom praw autorskich wymaga, aby uzyskać od nich zgodę na rozpowszechnianie ich utworów czy tworzenie na ich bazie utworów zależnych. Duże kłopoty pojawiają się jednak wtedy, kiedy chcąc skorzystać z~chronionego utworu nie możemy skontaktować się z~właścicielem praw do niego. Kontakt i~uzyskanie odpowiedniej zgody nie jest możliwe choćby i~dlatego, że właściciele praw autorskich są po prostu nieznani. Po kilkudziesięciu latach od śmierci twórcy bywa czasem niemożliwe ustalenie jego autorstwa lub nawet rocznej daty zgonu. Ponieważ autorskie prawa majątkowe są dziedziczone, dodatkowym wyzwaniem bywa prawidłowe także ustalenie spadkobierców. W~takim przypadku mamy do czynienia z~utworami osieroconymi (ang.
\slowoobce{orphan works}).} \akap{Status utworów osieroconych w~polskim prawie jest nieuregulowany — samo pojęcie nie pojawia się w~ustawie.
\wyroznienie{Osierocenie} uniemożliwia legalne korzystanie z~wielu utworów — trudno się na to zdecydować w~perspektywie potencjalnych roszczeń ze strony podmiotów praw autorskich:}
\dlugicytat{ \akap{Filmoteka Narodowa stwierdziła w~swoim stanowisku, że do czasu uregulowania stanu prawnego tzw. „dzieł osieroconych” w~swoich zbiorach nie będzie udostępniała ich chętnym właśnie ze względu na ryzyko niespodziewanych roszczeń}
} \akap{Wielkie kłopoty z~utworami osieroconymi mają także polskie biblioteki cyfrowe. Posiadając je w~swoich zbiorach w~formie fizycznych egzemplarzy wciąż nie mogą udostępniać ich online. Postawy bibliotek nie zmienia nawet to, że dominująca większość tych utworów (książek, czasopism, albumów itp.) nie jest dziś już źródłem jakiegokolwiek zysku i~nie ma znaczenia komercyjnego (często nie jest w~ogóle dostępna na rynku komercyjnym).}
\akap{Problem dzieł osieroconych ma charakter międzynarodowy — w~przypadku niektórych kategorii utworów (filmy czy muzyka) odsetek utworów o~takim nieznanym statusie jest bardzo wysoki. Dotyczy to szczególnie utworów opublikowanych w~czasie lub po wielkich konfliktach zbrojnych, podczas których niszczono archiwa, biblioteki i~dokumentacje gromadzone w~urzędach i~firmach, a~miliony ludzi traciły swoją oficjalną tożsamość — przepadały dokumenty osobiste, akta administracyjne, wiele osób zaginęło bez wieści.}
\akap{Rozwiązanie problemu dzieł osieroconych wymaga odejścia od przeświadczenia, że ochrona prawa autorskiego obowiązuje w~sposób bezwzględny i~chronione są przede wszystkim interesy autora i~właściciela autorskich praw majątkowych. Interes publiczny wymaga jak najszerszego dostępu do dóbr kultury, bez względu na prawne zawirowania czy jedynie wirtualne korzyści dla domniemanych posiadaczy praw autorskich. Problem ten pozostaje nierozstrzygnięty i~znacząco utrudnia pracę bibliotek i~archiwów cyfrowych — nawet przy zachowaniu szczególnej staranności ustalenie autorstwa (a tym samym zakresu ochrony prawa autorskich) może nie być możliwe.}
\akap{Unia Europejska przyjęła niedawno dyrektywę, która ma na celu ułatwienie bibliotekom i~archiwom udostępnianie w~Internecie utworów osieroconych. Póki co niektóre instytucje GLAM (\wyroznienie{Galleries, Libraries, Archives \& Museums}) próbują radzić sobie z~problemem utworów osieroconych na własną rękę:}
\dlugicytat{ \akap{Serwis Flickr w~ramach projektu Flickr Commons zachęca instytucje GLAM do publikacji utworów osieroconych z~informacją, że ograniczenia wynikające z~praw autorskich są nieznane (\slowoobce{no known copyright restrictions}). W~projekcie tym bierze udział m.in. biblioteka Kongresu.}
} \akap{Zastanawiając się nad skorzystaniem z~utworu osieroconego warto zawsze zastanowić się, jakie jest realne ryzyko ewentualnych roszczeń ze strony właściciela praw autorskich. Jeśli jest nieznany i~nie ma większych szans na to, żeby został ostatecznie zidentyfikowany, można zaryzykować. Na tej zasadzie utwory osierocone z~archiwów i~bibliotek z~całego świata publikowane są w~projekcie Flickr Commons.}
\naglowekpodrozdzial{FAQ} \listanum{ \punkt{ \akap{ \wyroznienie{Czy formą zapezpieczenia się przed ewentualnymi konsekwencjami publikacji utworów osieroconych jest poinformowanie ewentualnych właścicieli praw autorskich o~tym, że mogą zgłosić swoje prawa do nich. Można to zrobić np. w~formie informacji na stronie WWW, na której publikowane są takie materiały…}
} \akap{Jest to na pewno sposób na pokazanie woli dotarcia do właściciela praw autorskich, z~formalnego punktu widzenia nie sprawia to jednak, że upublicznienie utworu osieroconego staje się legalne. Tylko właściciel autorskich praw majątkowych może zezwolić na wykorzystanie utworu.}
} } \naglowekrozdzial{9. Dozwolony użytek prywatny} \akap{Autorskie prawa majątkowe ograniczają naszą swobodę korzystania z~utworu. Na szczęście w~prawie autorskim znajdują się przepisy pozwalające na korzystanie z~utworów na różne sposoby bez konieczności uzyskiwania zgody uprawnionych osób (właścicieli praw majątkowych), choć obwarowane jest to szeregiem ograniczeń. Te swobody użytkowników nazywamy
\wyroznienie{dozwolonym użytkiem}. Uzyskujemy je bez względu na wolę uprawnionych, ponieważ takie ograniczenie monopolu autorskiego zapisane jest bezpośrednio w~ustawie. Dozwolony użytek dzieli się na dozwolony użytek prywatny i~publiczny.}
\naglowekpodrozdzial{Formy dozwolonego użytku} \akap{Dozwolony użytek osobisty pozwala każdemu korzystać na własne potrzeby z~rozpowszechnionego już chronionego utworu. Rozciąga się on także na osoby pozostające z~nami w~relacjach rodzinnych i~towarzyskich. Przykłady dozwolonego użytku osobistego ilustruje poniższa tabelka:}
\par{} \vspace{1em} { \raggedright{}
\begin{tabularx}{\textwidth{}}{|X|X|X|X|X|}
\hline{} Przykłady dozwolonego użytku osobistego\\\hline Pobranie pliku z~muzyką ze strony WWW\\\hline
Pożyczenie książki znajomemu\\\hline Skserowanie książki na potrzeby własne bądź znajomych\\\hline
Wykonanie kopii płyty CD do samochodu\\\hline Puszczanie muzyki na urodzinach\\\hline
Odwiedzenie strony WWW\\\hline
\end{tabularx}
} \vspace{1em} \akap{Jak widać, dozwolony użytek osobisty działa bez względu na formę (postać) utworu: nieważne jest, czy mamy do czynienia z~fizycznym egzemplarzem (książka, płyta) czy z~plikiem cyfrowym. Istotnym ograniczeniem dozwolonego użytku osobistego jest zawężenie jego zasięgu do nas samych oraz bliskiego kręgu znanych nam osób.}
\akap{Spróbujmy zanalizować taki przykład:} \dlugicytat{ \akap{Jakub kupił audiobooka — adaptację ulubionej książki — w~jednej z~internetowych księgarni. Ponieważ nagranie bardzo mu się się podobało zrecenzował je na swoim blogu i~wszystkim chętnym udostępnił pliki z~nagraniami.}
} \akap{Działania Jakuba wykraczają poza dozwolony użytek prywatny. Bez problemu mógłby wysłać pliki audiobooka do swoich dobrych znajomych emailem. Może także skopiować je na płytę CD do odsłuchania w~samochodzie. Może także skopiować je na pamięci USB, które da w~prezencie gwiazdkowym wszystkim swoim krewnym. Jednak udostępnienie utworów wszystkim użytkownikom internetu wykracza poza zakres dozwolonego użytku osobistego.}
\akap{W wielu kampaniach medialnych organizowanych przez instytucje publiczne czy producentów filmowych lub muzycznych, skierowanych przeciwko ściąganiu plików z~internetu, brakuje często informacji o~podstawowych prawach, jakie mamy dzięki istniejącym w~ustawie o~prawie autorskim zapisom o~dozwolonym użytku prywatnym. Warto mieć świadomość że nie zawsze to, co w~mediach określane jest jako nielegalne, rzeczywiście łamie zasady prawa.}
\naglowekpodrozdzial{FAQ} \listanum{ \punkt{ \akap{ \wyroznienie{Czy dozwolony użytek obejmuje też utwory z~domeny publicznej?}
} \akap{Nie. Utwory, co do których wygasła ochrona autorskich praw majątkowych mogą być swobodnie wykorzystywane — nie ma tu ograniczenia do kręgu znajomych nam osób.}
} \punkt{ \akap{ \wyroznienie{Czy w~ramach dozwolonego użytku mogę ściągnąć z~internetu program komputerowy?}
} \akap{Nie. W~ustawie o~prawie autorskim znajduje się wyraźne wyłączenie dozwolonego użytku prywatnego w~stosunku do programów komputerowych. Ściąganie z~internetu programów komputerowych możliwe jest tylko wtedy, gdy zostały one tam umieszczone do pobrania przez samych uprawnionych (twórców, producentów, oficjalnych dystrybutorów itp.) lub za ich zgodą (np. wolne oprogramowanie, ale też oprogramowanie freeware czy shareware).}
} \punkt{ \akap{ \wyroznienie{Korzystam z~torrentów i~w~ten sposób ściągam muzykę. Czy działam legalnie?}
} \akap{Nie, ponieważ korzystanie z~sieci torrent wymaga — przy pobieraniu — równoczesne udostępnianie plików. O~ile pobieranie plików mieści się w~granicach dozwolonego użytku, to ich publiczne rozpowszechnianie już nie.}
} } \naglowekrozdzial{10. Dozwolony użytek publiczny} \akap{Dozwolony użytek może mieć także charakter publiczny. O~ile dozwolony użytek prywatny (osobisty) ograniczał się jedynie do kręgu osób pozostających w~relacjach towarzyskich czy rodzinnych, dozwolony użytek publiczny zezwala na nieodpłatne (co do zasady) korzystanie z~już rozpowszechnionego utworu na różne sposoby, które powodują udostępnienie go nieograniczonej publiczności. Przykłady dozwolonego użytku publicznego pokazuje poniższa tabela:}
\par{} \vspace{1em} { \raggedright{}
\begin{tabularx}{\textwidth{}}{|X|X|X|X|X|}
\hline{} Przykłady dozwolonego użytku publicznego\\\hline Biblioteka wypożyczająca książki\\\hline
Przytoczenie fragmentu cudzej książki we własnym utworze np. w~celu wyjaśnienia poruszonej kwestii (cytat)\\\hline
Wyświetlenie ekranizacji lektury szkolnej na lekcji języka polskiego poświęconej tej lekturze\\\hline
Publiczne wykonanie wiersza współczesnego poety podczas szkolnej akademii\\\hline
Wykonanie i~rozpowszechnienie w~internecie fotografii pomnika wystawionego na stałe w~miejscu publicznym\\\hline
\end{tabularx}
} \vspace{1em} \akap{Zasady dozwolonego użytku publicznego są nieco bardziej skomplikowane od zasad dozwolonego użytku osobistego. Postaramy się poniżej przedstawić najważniejsze z~nich.}
\naglowekpodrozdzial{Cytat} \akap{Przepis o~dozwolonym cytacie stanowi, że wolno nam przytaczać fragmenty utworów w~ramach osobiście tworzonego dzieła. Jeśli mamy do czynienia z~drobnym utworem (np. krótkim wierszem) możemy nawet przytoczyć go w~całości. Dłuższe utwory można cytować tylko we fragmentach. Niestety, prawo nie daje nam definicji „drobnego utworu”.}
\akap{Cytowanie poddane jest pewnym ograniczeniom. Możemy we własnej pracy przytoczyć całość bądź fragment innego dzieła (wszystko jedno, czy jest to tekst, zdjęcie czy film), ale tylko wtedy, gdy jest to niezbędne do
\wyroznienie{wyjaśnienia jakiegoś problemu, w~związku z~analizą krytyczną dzieła albo w~celu edukacyjnym}.}
\dlugicytat{ \akap{W literaturoznawczej książce naukowej autor przytacza obszerne fragmenty współczesnej prozy w~celu ilustracji stawianych przez siebie tez.}
} \akap{Przepisy ustawy mówią jeszcze o~tym, że stosowanie cytatu może wynikać z~praw gatunku twórczości:}
\dlugicytat{ \akap{W wierszu opublikowanym w~czasopiśmie literackim pojawiają się fragmenty wierszy innych poetów}
} \akap{Wiele gatunków twórczości opiera się na wykorzystaniu cudzych utworów — w~formie pośredniego bądź bezpośredniego nawiązania. Istotą parodii, pastiszu czy karykatury jest odwołanie się do znanych już odbiorcy cudzych utworów. Takie formy twórczości co do zasady mogą posługiwać się cytatem.}
\akap{Ważne jest, żeby cytat był odpowiednio oznaczony. Stosowanie go wymaga czytelnego oznaczenia, że korzysta się z~cudzego utworu. Dlatego posługując się cytatem informujemy zawsze o~\wyroznienie{autorze}
oraz \wyroznienie{źródle} cytowanego fragmentu. Jeśli jego autor nie jest znany, należy również o~tym poinformować. Brak takiej informacji oznacza plagiat. Źródło należy oznaczyć w~sposób, który umożliwi jednoznaczną identyfikację utwory, z~którego skorzystaliśmy. W~przypadku publikacji naukowych jest to zazwyczaj pełen adres bibliograficzny, ale dla innych publikacji zazwyczaj wystarczy podać tytuł utworu lub link do strony internetowej.}
\akap{Cytaty stosować można nie tylko w~utworach istniejących w~postaci tekstowej oraz nie tylko w~celach naukowych. Dla przykładu, są one jednym z~najważniejszych elementów wariacji jako gatunku muzycznego. Muzyk sięga po fragmenty innych utworów, którymi wzbogaca własną kompozycję. Także remiks jako gatunek budowany jest na cytatach. Cytaty mogą pojawiać się także w~filmach:}
\dlugicytat{ \akap{W narrację filmu dokumentalnego poświęconego historii polskiego kina włączono krótkie fragmenty „Człowieka z~marmuru” Andrzeja Wajdy, komentowane bezpośrednio przez lektora.}
} \akap{Istnieje nawet specjalny gatunek twórczości filmowej „\slowoobce{found footage}” polegający na tworzeniu filmów bez użycia kamery — wyłącznie z~fragmentów innych produkcji, amatorskich lub archiwalnych nagrań itp. Niestety, ponieważ granice dozwolonego użytku nie są określone w~sposób jednoznaczny, bardzo często trudno jest samodzielnie stwierdzić, czy nie naruszamy przypadkiem prawa. Ten brak jasności przepisów szczególnie dotkliwy jest np. w~przypadku filmów dokumentalnych, ponieważ bardzo często producenci nie chcąc wikłać się w~procesy sądowe poszukują zgody twórców i~innych uprawnionych także w~przypadkach, w~których przesłanki dozwolonego cytowania mogłyby zostać uznane za spełnione.}
\naglowekpodrozdzial{Biblioteki, archiwa i~szkoły} \akap{Biblioteki, archiwa i~szkoły zostały w~specjalny sposób potraktowane przez twórców systemu prawa autorskiego. W~odróżnieniu od innych organizacji czy osób fizycznych mają one prawo — w~ramach dozwolonego użytku publicznego — nieodpłatnie udostępniać egzemplarze rozpowszechnionych, chronionych utworów. Co więcej, mają prawo wykonywać również kopie egzemplarzy utworów już rozpowszechnionych, jeśli celem jest uzupełnienie czy ochrona własnych zbiorów. Dzięki temu np.:}
\dlugicytat{ \akap{Biblioteki uczelniane mogą digitalizować współczesne podręczniki akademickie i~udostępniać je w~ramach wewnętrznej sieci bibliotecznej.}
} \dlugicytat{ \akap{Biblioteka wypożycza książkę z~innej instytucji, sporządza jej kopię i~udostępnia w~ramach standardowych wypożyczeń.}
} \akap{Chociaż utwory objęte autorskim prawem majątkowym nie mogą być przez bibliotekę publikowane w~otwartym internecie, specjalny przepis w~ustawie o~prawie autorskim umożliwia bibliotece rozpowszechnianie ich za pośrednictwem końcówek systemu informatycznego (terminali) znajdujących się na terenie tych instytucji. Cele takiego rozpowszechniania muszą być jednak „badawcze” lub „poznawcze”, co powoduje wątpliwości w~jaki sposób i~jakie utwory można tak udostępniać (np. czy zapis ten dotyczy publikacji o~charakterze rozrywkowym).}
\naglowekpodrozdzial{Działalność naukowa i~dydaktyczna} \akap{W ramach dozwolonego użytku publicznego instytucje naukowe i~oświatowe mogą korzystać z~już rozpowszechnionych utworów w~celach dydaktycznych i~do prowadzenia własnych badań. Zasada ta dotyczy także utworów, co których dostęp uzyskano przez internet:}
\dlugicytat{ \akap{Nauczyciel chce prowadzić lekcję dotyczącą historii PRL. W~ramach zajęć wykorzysta objęte prawami wyłącznymi fotografie oraz kroniki filmowe, dostępne w~internecie oraz na płytach CD dołączanych do tygodników.}
} \akap{Na tej samej zasadzie nauczyciel może na lekcji wyświetlić pełną wersję filmu, o~ile tylko widzami będą uczniowie bezpośrednio zaangażowani w~lekcję, a~projekcja służyć będzie celom dydaktycznym.}
\naglowekpodrozdzial{Prawo przedruku} \akap{Ta zasada dozwolonego użytku publicznego dotyczy prasy, radia, telewizji i~pozwala na dalsze rozpowszechnianie opublikowanych już materiałów takich jak: sprawozdania o~aktualnych wydarzeniach, aktualne artykuły na tematy polityczne, gospodarcze lub religijne, aktualne wypowiedzi i~fotografie reporterskie. Prawo przedruku pozwala też na publikowanie treści mów wygłoszonych na publicznych zebraniach i~rozprawach oraz krótkich streszczeń rozpowszechnionych utworów. Przepisy ustawy wymieniają dokładnie warunki, na których skorzystać można z~prawa do przedruku - w~niektórych przypadkach oznacza to np. konieczność zapłaty wynagrodzenia, w~innych twórcy mogą np. zablokować możliwość republikacji. Istnieją także kontrowersje dotyczące tego, czy prawo przedruku dotyczy także mediów internetowych (czy mogą być one traktowane jak prasa).}
\akap{Warto zwrócić uwagę, że prawo przedruku dotyczy utworów dokumentujących lub odnoszących się do wydarzeń aktualnych, które są akurat obiektem publicznego zainteresowania. Trudno więc oczekiwać, że ta zasada dozwolonego użytku pozwolinp. na republikowanie tekstów naukowych. Trudno w~kilku zdaniach opisać wszystkie elementy prawa przedruku — przed skorzystaniem z~tej formy dozwolonego użytku na pewno warto poznać ją głębiej.}
\naglowekpodrozdzial{Architektura, pomniki i~wystawy} \akap{Nie możemy zapominać, że utwory to nie tylko kompozycje muzyczne, teksty dziennikarskie czy literackie, fotografie i~filmy. Ochroną prawa autorskiego objęte są także pomniki, rzeźby, architektura. Są one stałym elementem przestrzeni miejskiej. Zrobienie i~umieszczenie online zdjęcia współczesnej rzeźby obecnej stale w~publicznie dostępnym miejscu (na ogólnie dostępnych drogach, ulicach, placach) jest formą korzystania z~takiego utworu. Dozwolony użytek pozwala nam rozpowszechniać wizerunki takich obiektów bez pytania o~zgodę twórców i~wypłacania im wynagrodzenia. Możemy to robić nawet w~celach komercyjnych. Nie możemy jednak kopiować rzeźb czy pomników do tego samego użytku, czyli np. w~celu umieszczenia tych kopii w~innym miejscu (na przykład we własnym ogródku). Takie działanie wymaga już zgody uprawnionego.}
\akap{Sprawa komplikuje się w~przypadku muzeów czy galerii. Tutaj prawo do rozpowszechniania fotografii czy reprodukcji obiektów ograniczone jest jedynie do katalogów czy publikacji (folderów, ulotek, plakatów) przygotowanych z~myślą o~promocji tych utworów. Można też rozpowszechniać takie utwory w~mediach, jednak jedynie w~celach informacyjnych}
\akap{Oczywiście jeśli fotografowane obiekty są w~domenie publicznej, nie ma problemu z~rozpowszechnianiem ich wizerunków. Warto jednak zwrócić uwagę na to, że nawet jeśli nie poszczególne obiekty (np. średniowieczne miniatury), to sam układ ekspozycji oraz opis pokazywanych w~jej ramach elementów może mieć charakter twórczy i~podlegać ochronie prawa autorskiego.}
\naglowekpodrozdzial{Dla osób niepełnosprawnych} \akap{W ramach dozwolonego użytku publicznego można sporządzać i~rozpowszechniać kopie oraz adaptacje utworów w~celu ułatwienia czy umożliwienia osobom niepełnosprawnym zapoznania się z~nimi. Działanie takie nie może mieć zarobkowego charakteru i~wynikać musi wprost z~określonej niepełnosprawności:}
\dlugicytat{ \akap{Dozwolony użytek publiczny pozwala legalnie przygotować audiobookową adaptację powieści dla osób niewidomych i~niedowidzących.}
} \akap{Dozwolony użytek publiczny jest dość obszerną kategorią w~ramach systemu prawa autorskiego, dlatego zdecydowaliśmy się przedstawić tylko niektóre, najważniejsze naszym zdaniem jego elementy.}
\naglowekpodrozdzial{FAQ} \listanum{ \punkt{ \akap{ \wyroznienie{Czy w~ogóle mogę fotografować w~muzeach?}
} \akap{Decyzją Urzędu Ochrony Konkurencji i~Konsumentów zakaz fotografowania w~muzeach wpisany do regulaminów tych instytucji jest nieważny. Muzea mogą jednak zakazywać używania lampy błyskowej, która może mieć negatywny wpływ na prezentowane w~ramach ekspozycji obiekty. Prawo do fotografowania utworu nie oznacza jednak automatycznego prawa do rozpowszechniania takiej fotografii. Można bez ograniczeń rozpowszechniać własne fotografie utworów z~domeny publicznej, ale w~przypadku fotografii utworów chronionych należy albo uzyskać zgodę uprawnionych do tych utworów, albo ograniczyć korzystanie do wąskiego zakresu określonego przepisu o~dozwolonym użytku.}
} } \naglowekrozdzial{11. Utwory zależne} \akap{W jednym z~pierwszych rozdziałów pisaliśmy o~pojęciu utworu, który ma podstawowe znaczenie dla rozumienia zasad działania prawa autorskiego. Opisaliśmy tam w~kilku zdaniach podział na utwory pierwotne i~zależne. Przypomnijmy:}
\dlugicytat{ \akap{Utworem pierwotnym jest np. anglojęzyczny oryginał
\tytuldziela{Władcy Pierścieni} J.R.R. Tolkiena. Utworem zależnym jest polskie tłumaczenie tej powieści, którego rozpowszechnianie wymaga zgody właścicieli praw autorskich do utworu pierwotnego.}
} \akap{Oczywiście utwory zależne to nie tylko tłumaczenia ważnych dzieł kultury. Skoro każdy z~nas jest twórcą, każdy też może tworzyć utwory zależne. Na przykład w~taki sposób:}
\dlugicytat{ \akap{Asia uczy się angielskiego i~lubi oglądać filmy dokumentalne. Łączy te dwie aktywności przygotowując polskie napisy do filmów dostępnych na YouTube.}
} \akap{Asia tworzy i~rozpowszechnia utwory zależne, na co powinna uzyskać zezwolenie ze strony właścicieli praw autorskich do tłumaczonego materiału.}
\akap{System utworów zależnych może być bardzo skomplikowany — na bazie istniejących już utworów zależnych powstają przecież kolejne. Zobaczmy, ile utworów zależnych może powstać na bazie tylko jednej książki:}
\par{} \vspace{1em} { \raggedright{}
\begin{tabularx}{\textwidth{}}{XXXXX}
Oryginał książki w~języku angielskim\\Scenariusz filmowy w~języku angielskim na bazie książki\\Polskie tłumaczenie listy dialogowej w~wydaniu DVD filmu\\Komiks na bazie polskiego tłumaczenia listy dialogowej i~fotosów z~filmu\\
\end{tabularx}
} \vspace{1em} \par{} \vspace{1em} { \raggedright{}
\begin{tabularx}{\textwidth{}}{XXXXX}
Oryginał książki w~języku angielskim\\Scenariusz filmowy w~języku angielskim na bazie książki\\Gra komputerowa z~listą dialogową na bazie scenariusza filmu\\Tłumaczenie listy dialogowej gry\\
\end{tabularx}
} \vspace{1em} \akap{Na każdym etapie tego procesu twórcy opracowania, przeróbki czy adaptacji uzyskać muszą odpowiednią zgodę na rozpowszechnianie utworów zależnych ze strony uprawnionego do oryginalnego utworu (utworów):}
\akap{Dobrym przykładem problemów z~utworami zależnymi jest próba stworzenia podręcznika pokazującego najważniejsze kierunki sztuki współczesnej, ilustrowanego odpowiednimi reprodukcjami i~fragmentami recenzji krytycznych. Aby opublikować książkę jej wydawca musi pozyskać odpowiednie zgody. Nie jest to prosty proces: musi przede wszystkim ustalić, kto ma prawo udzielenia zgody na umieszczenie reprodukcji czy tekstu w~albumie, następnie zdobyć zgodę oraz ustalić i~wypłacić wynagrodzenie. W~przypadku kilkudziesięciu czy kilkuset utworów jest to spore wyzwanie.}
\naglowekrozdzial{12. Organizacje Zbiorowego Zarządzania} \akap{Niektóre formy działalności siłą rzeczy oznaczają konieczność zarządzania prawami do setek i~tysięcy utworów — na przykład taki problem ma każda stacja radiowa. Sposobem rozwiązania tego problemu są Organizacje Zbiorowego Zarządzania (OZZ), które zajmują się głównie udzielaniem zezwoleń na wykorzystanie utworów niejako hurtem. Podstawą działania OZZ jest albo ustawa (w niektórych przypadkach przepisy nakazują korzystać z~utworów za pośrednictwem OZZ), albo umowa (twórcy podpisując z~OZZ umowy przekazują im część praw do zarządzania własnymi utworami i~zgadzają się też, że to za pośrednictwem OZZ otrzymywać będą wynagrodzenie za wykorzystanie swojej twórczości) albo działanie bezumowne (tzw. prowadzenie cudzych spraw bez zlecenia, podejmowane w~przeświadczeniu, że nieuchwytny twórca chciałby pobierać wynagrodzenie za swoje utwory wg stawek OZZ, a~nie np. zezwolić na ich nieodpłatne wykorzystanie bądź w~ogóle zakazać określonego wykorzystania).}
\dlugicytat{ \akap{Stacja radiowa puszcza w~ciągu doby setki piosenek objętych autorskim prawem majątkowym. Rozliczanie takiego korzystania z~utworów bezpośrednio z~każdym z~twórców nie jest łatwe. Internet pozwala co prawda rezygnować z~pośredników, ale prawo autorskie wciąż zobowiązuje nadawców do uzyskiwania zezwoleń przez OZZ (chyba że uzyskają od twórcy pisemne zrzeczenie się tego pośrednictwa). Zapisany w~ustawie wymóg umowy w~formie pisemnej faktycznie uniemożliwia skuteczne wykorzystywanie internetowych systemów rozliczeń.}
} \akap{OZZ w~Polsce to m.in. Stowarzyszenie Artystów Wykonawców Utworów Muzycznych i~Słowno\dywiz{}Muzycznych (SAWP), Związek Artystów Scen Polskich (ZASP), Stowarzyszenie Filmowców Polskich (SFP), Związek Producentów Audio\dywiz{}Video (ZPAV) czy Stowarzyszenie Autorów ZAiKS. W~sumie jest to kilkanaście różnych Organizacji Zbiorowego Zarządzania, które reprezentują różnych uprawnionych i~mają różny zakres działania. Na przykład Stowarzyszenie Kopipol pobiera opłaty od producentów papieru, czystych płyt CD, twardych dysków i~innych nośników informacji — bo być może na części z~nich będą zapisane utwory objęte monopolem prawnoautorskim.}
\akap{Niestety, opłacenie jednej organizacji zbiorowego zarządzania nie oznacza, że wypełniliśmy wszystkie obowiązki. Na przykład posiadając zakład fryzjerski, w~którym puszczamy klientom muzykę z~płyt, jesteśmy zobowiązani osobno do opłat na rzecz twórców utworów, osobno na rzecz producentów — i~tak dalej. To oznacza, że możemy spodziewać się wizyt inspektorów wielu różnych OZZ, i~każda nałoży na nas swoje opłaty.}
\akap{Warto pamiętać, że płacenie OZZ z~tytułu wykorzystania utworów poza zakresem dozwolonego użytku nie jest konieczne, jeśli korzystamy wyłącznie z~utworów publikowanych na wolnych licencjach}
\naglowekrozdzial{13. Pola eksploatacji} \akap{Pole eksploatacji to pojęcie wskazujące różne sposoby korzystania z~utworów. W~czasach mediów analogowych pola eksploatacji dało się dość dobrze zdefiniować: czym innym była publikacja książek, a~czym innym spektakl teatralny. Dziś, w~czasach komunikacji poprzez media cyfrowe, podział na pola eksploatacji powoli traci sens, ale ciągle jest obecny w~prawie. W~systemie polskiego prawa autorskiego nie ma pełnego katalogu pól eksploatacji — w~ustawie znaleźć można kilkanaście przykładów: technika drukarska, reprograficzna, zapis magnetycznego, technika cyfrowa, wprowadzanie do obrotu, użyczenie lub najem oryginału albo egzemplarzy, publiczne wykonanie, wystawienie, wyświetlenie, odtworzenie oraz nadawanie i~reemitowanie.}
\akap{Szczególnym przykładem pola eksploatacji jest „publiczne udostępnianie utworu w~taki sposób, aby każdy mógł mieć do niego dostęp w~miejscu i~w~czasie przez siebie wybranym”, które w~praktyce oznacza wszystkie media cyfrowe, a~w~szczególności Internet. To bardzo sprytna definicja ujmująca istotę mediów cyfrowych. Media analogowe albo są ograniczone poprzez miejsce (żeby kupić książkę musimy pójść do księgarni), albo poprzez czas (piosenki w~radio możemy posłuchać tylko w~chwili, gdy jest nadawana). Sieciowe media cyfrowe zapewniają nam swobodę wyboru i~czasu, i~miejsca korzystania z~utworu.}
\akap{Niniejsza tabela pokazuje przykłady zastosowania wybranych pól eksploatacji:}
\par{} \vspace{1em} { \raggedright{}
\begin{tabularx}{\textwidth{}}{XXXXX}
\wyroznienie{pole eksploatacji}&\wyroznienie{przykład wykorzystania}\\technika drukarska&wydanie powieść w~formie drukowanej książki\\zapis magnetyczny&utrwalenie filmu na kasetach video\\dźwiękowy zapis cyfrowy&wydanie powieści w~formie audiobooka zapisanego w~plikach mp3\\publiczne wykonanie&zaśpiewanie piosenki podczas koncertu\\publiczne odtwarzanie&odtworzenie piosenki na dyskotece\\publiczne udostępnianie utworu w~taki sposób, aby każdy mógł mieć do niego dostęp w~miejscu i~w~czasie przez siebie wybranym&udostępnienie fotografii cyfrowych w~Internecie\\
\end{tabularx}
} \vspace{1em} \akap{Świadomość istnienia pól eksploatacji jest bardzo ważna, ponieważ zgodnie z~prawem jakakolwiek umowa o~wykorzystanie utworu musi być zawarta na określone pola eksploatacji. Nie można po prostu zawrzeć umowy na „wszystkie pola eksploatacji” — trzeba szczegółowo je wymienić. Na dodatek w~umowie wypisać można jedynie te pola eksploatacji, które są znane w~czasie podpisania umowy. Zapis taki jak ten:}
\dlugicytat{ \akap{Strona umowy, będąca właścicielem autorskich praw majątkowych, zezwala na wytwarzanie egzemplarzy utworu techniką drukarską, reprograficzną, zapisu magnetycznego, techniką cyfrową oraz wszystkimi innymi dostępnymi i~potencjalnymi technikami.}
} \akap{…nie będzie ważny w~zakresie w~jakim odnosi się do nieistniejących jeszcze pól eksploatacji. Użycie słowa
\wyroznienie{wszystkie} także jest problematyczne. Brak wyraźnego wymienienia pól eksploatacji powoduje dodatkowe komplikacje przy interpretacji zapisów umowy.}
\akap{W większości przypadków skutecznym rozwiązaniem tego problemu jest wymienienie wszystkich pól eksploatacji wymienionych w~ustawie, w~tym koniecznie używając formuły „\wyroznienie{publiczne udostępnianie utworu w~taki sposób, aby każdy mógł mieć do niego dostęp w~miejscu i~w~czasie przez siebie wybranym}”. Trudno bowiem wyobrazić sobie jakieś nowe medium cyfrowe, którego ta definicja nie obejmie.}
\naglowekrozdzial{14. Przeniesienie praw majątkowych} \akap{Korzystanie z~utworu poza zakresem dozwolonego użytku oznacza wejście w~zakres monopolu autorskiego. W~prawie autorskim istnieją dwa podstawowe sposoby zrobienia tego w~sposób legalny: uzyskanie licencji uprawnionego (umowa licencyjna) lub nabycie od niego autorskich praw majątkowych (umowa o~przeniesienie praw). Bardzo często twórcy i~przedsiębiorcy nie wiedzą, czym się one od siebie różnią, tymczasem różnice te są bardzo duże.}
\dlugicytat{ \akap{Autor powieści podpisał z~wydawnictwem umowę o~przeniesieniu praw autorskich. Oznacza to, że teraz wydawnictwo, a~nie autor posiada wskazane w~umowie prawa do tego tekstu.}
} \akap{Umowa o~przeniesienie praw majątkowych jest trochę jak umowa na sprzedaż mieszkania. Po jej podpisaniu jedyną osobą uprawnioną do korzystania z~utworu staje się nabywca. Pierwotny uprawniony po jej podpisaniu traci wskazane w~umowie prawa majątkowe i~i~nie może z~nich korzystać. W~praktyce umowa o~przeniesienie ma charakter ostateczny i~nieodwołalny, choć ustawa pozwala od niej odstąpić w~pewnych wyjątkowych przypadkach. Bardzo często umowa o~przeniesienie jest pierwszą propozycją, jaką wydawca podsuwa do podpisania twórcy (jest ona zdecydowanie korzystniejsza dla nabywcy i~daje mu duże większe bezpieczeństwo korzystania z~praw). Z~kolei twórcy po podpisaniu takiej umowy często dopiero po jakimś czasie odkrywają, że nie mogą podjąć decyzji o~samodzielnym rozpowszechnianiu swoich własnych utworów.}
\akap{Oczywiście, twórca nadal może korzystać z~praw osobistych, ponieważ te są niezbywalne. Jednak w~praktyce to osoba posiadająca prawa majątkowe kontroluje wykorzystanie utworu.}
\akap{Licencja jest zupełnie innym typem umowy. Umowa licencyjna oznacza udzielenie zgody na wykorzystanie utworu, ale prawa majątkowe nadal pozostają przy licencjodawcy. Dzięki temu może on na przykład umowę licencyjną… wypowiedzieć. Dlatego umowy licencyjne są zawsze bardziej ryzykowne dla licencjobiorców.}
\par{} \vspace{1em} { \raggedright{}
\begin{tabularx}{\textwidth{}}{XXXXX}
\wyroznienie{umowa o~przekazaniu praw}&\wyroznienie{umowa licencyjna}\\twórca traci określone prawa majątkowe do swojego utworu (na określonym polu eksploatacji)&upoważnia do korzystania z~tych praw na określonym polu eksploatacji\\strona umowy staje się podmiotem praw&strona umowy jest jedynie upoważniona do korzystania z~praw\\przeniesienie praw na danym polu eksploatacji to akt jednorazowy&można udzielić wielu licencji na danym polu eksploatacji, o~ile mają one charakter niewyłączny\\
\end{tabularx}
} \vspace{1em} \akap{Zarówno umowy licencyjne jak i~umowy o~przeniesienie praw podpisuje się na konkretne pola eksploatacji.}
\naglowekrozdzial{15. Licencje} \akap{Pewnie nie zdajecie sobie z~tego sprawy, ale z~licencji korzystamy codziennie:}
\dlugicytat{ \akap{Pracując na komputerze z~zainstalowanym systemem operacyjnym Windows korzystamy z~praw przysługujących nam w~ramach licencji, na którą zgodziliśmy się instalując oprogramowanie.}
} \akap{Licencja to po prostu sposób na udzielanie zgody na konkretne wykorzystanie utworu. W~odróżnieniu od umowy przeniesienia praw, korzystając z~licencji właściciel praw nie traci swoich przywilejów wynikających z~autorskiego prawa majątkowego.}
\dlugicytat{ \akap{Autor powieści podpisał z~wydawnictwem niewyłączną umowę licencyjną. Oznacza to, że wydawnictwo może korzystać z~tego tekstu w~określony umową sposób i~za wynagrodzeniem. Pisarz nie traci swoich praw, a~jedynie akceptuje zapisane w~umowie działania wydawnictwa.}
} \naglowekpodrozdzial{Umowy licencyjne} \akap{Licencjodawca w~ramach umowy licencyjnej zezwala licencjobiorcy na korzystanie z~utworu. Strony umowy, a~często sam licencjodawca jednostronnie określa, kto może z~tych praw korzystać i~na jakich zasadach. Nigdy jednak nie może udzielić komuś więcej praw niż sam posiada.}
\akap{Umowa licencyjna niewyłączna może mieć dowolną formę, także ustną. Oczywiście, forma usta nie zawsze jest dobrym rozwiązaniem, ponieważ zawarcie i~zakres takiej umowy jest trudne do udowodnienia i~wyegzekwowania w~przypadku sporu pomiędzy stronami. Natomiast w~codziennych sytuacjach, pozbawionych dużego ryzyka, jest ona całkowicie wystarczająca. Przykładem bardzo prostej licencji zawartej ustnie jest poniższa wymiana zdań:}
\dlugicytat{ \akap{— Mogę wrzucić twoją fotkę na swojego bloga?}
\akap{— Jasne.} } \akap{Dużo bezpieczniejsza jest oczywiście licencja zawarta w~formie pisemnej umowy, gdzie strony wspólnie ustalają i~spisują zasady.}
\akap{Licencja może mieć charakter \wyroznienie{wyłączny} lub \wyroznienie{niewyłączny}. Niewyłączność licencji oznacza, że licencjobiorca nie ma wyłączności na korzystanie z~utworu w~zakresie opisanym w~umowie. Twórca może udzielić wielu licencji niewyłącznych na tym samym polu eksploatacji:}
\dlugicytat{ \akap{Fotograf przygotował dla lokalnego samorządu zdjęcia zabytków miasteczka, w~którym mieszka, na potrzeby strony internetowej urzędu. W~umowie licencyjnej z~władzami gminy nie znalazło się postanowienie o~jej wyłącznym charakterze. Dzięki temu fotograf mógł podpisać kolejną umowę na rozpowszechnianie swoich zdjęć — tym razem na stronie drużyny harcerskiej.}
} \akap{Jest to standardowy sposób licencjonowania — zgodnie z~przepisami zapisu o~niewyłączności licencji nie trzeba nawet wprowadzać do tekstu umowy.}
\akap{Inaczej jest w~przypadku licencji wyłącznej — tutaj konieczne jest wyraźne stwierdzenie faktu takiego licencjonowania. Umowa licencji wyłącznej musi też bezwzględnie mieć formę pisemną, gdyż inaczej jest nieważna. Gdyby fotograf z~powyższego przykładu zdecydował się na podpisanie z~samorządem umowy na licencję wyłączną, to nie mógłby już udzielić takiej zgody harcerzom.}
\akap{Chociaż standardowa licencja umowna jest bardziej elastyczna niż umowa o~przeniesienie praw, to ma swoje wady. Wciąż konieczne jest nawiązanie kontaktu między właścicielem praw autorskich i~osobą, przedsiębiorstwem lub instytucją chcącą skorzystać z~utworu. Czasochłonne może być też ustalenie warunków umowy i~zredagowanie jej treści. To wszystko komplikuje korzystanie z~utworu:}
\dlugicytat{ \akap{Ania chciała przygotować artykuł na temat sytuacji kobiet w~krajach islamskich. Ilustracją tekstu miało być zdjęcie z~jednego z~protestów w~Jemenie, które Ania znalazła przez Google Image Search na jednym z~lokalnych blogów. Niestety, kilkukrotne próby mailowego skontaktowania się z~jego autorem z~prośbą o~licencję nie przyniosły rezultatu i~ostatecznie artykuł pozostał bez ilustracji.}
} \akap{Na szczęście umowy licencyjne nie zawsze muszą funkcjonować według tego modelu.
\wyroznienie{Wolne licencje} są licencjami udzielanymi z~góry: twórca w~chwili publikacji decyduje o~sposobie licencjonowania i~informuje o~tym wszystkich potencjalnych licencjobiorców. Gdyby jemeński bloger opublikował swoje zdjęcie na wolnej licencji, Ania mogłaby legalnie wykorzystać je w~gazetce szkolnej od razu, bo zgoda autora zdjęć byłaby z~góry zapisana już w~treści tej licencji.}
\naglowekpodrozdzial{FAQ} \listanum{ \punkt{ \akap{ \wyroznienie{Czy mogę stworzyć własną licencję?}
} \akap{Każdy może stworzyć własną licencję. Należy jednak zadbać, aby jej treść była zgodna z~obowiązującym prawem i~odpowiednio czytelnie określała prawa i~obowiązki licencjobiorcy oraz ograniczenia w~korzystaniu z~utworu.}
} \punkt{ \akap{ \wyroznienie{Czy można jeden utwór licencjonować za pomocą wielu licencji wyłącznych?}
} \akap{Tak, jeden utwór można licencjonować za pomocą wielu licencji wyłącznych. Licencje te dotyczyć muszą jednak różnych pól eksploatacji. W~treści licencji obowiązkowo musi znaleźć się także zapis o~jej wyłącznym charakterze.}
} \punkt{ \akap{ \wyroznienie{Czego mogą dotyczyć ograniczenia w~umowie licencyjnej?}
} \akap{Ograniczenia korzystania z~utworu, opisane w~umowie licencyjnej, dotyczyć mogą:}
\listapunkt{ \punkt{czasu obowiązywania licencji — jeśli nie zostanie to wyraźnie określone, licencja jest zawarta na okres 5~lat (udzieloną na okres dłuższy, po upływie tego terminu, uważa się za udzieloną na czas nieokreślony),}
\punkt{terytorium — o~ile nie zostanie to wyraźnie oznaczone w~treści umowy, licencja jest ważna na terytorium państwa, w~którym jest siedziba licencjobiorcy,}
\punkt{pól eksploatacji,} \punkt{wyłączności lub niewyłączności licencji,}
\punkt{możliwości sublicencjonowania} \punkt{wynagrodzenia dla posiadacza autorskich praw majątkowych, który licencjonuje wykorzystanie utworu.}
} } \punkt{ \akap{ \wyroznienie{Jak korzystać z~licencji, kiedy autorskie prawa majątkowe do utworu posiada kilka osób?}
} \akap{W takim przypadku osoby te muszą wystąpić wspólnie jako jedna strona umowy. Nie ma żadnych ograniczeń związanych z~liczbą podmiotów (osób czy instytucji) mogących stać się licencjonobiorcami.}
} \punkt{ \akap{ \wyroznienie{Czy każda licencja nakłada na licencjobiorcę obowiązek wypłaty wynagrodzenia posiadaczowi praw autorskich?}
} \akap{Przepisy prawa autorskiego mówią, że twórcy przysługuje prawo do wynagrodzenia. Nie jest ono jednak obligatoryjne i~zależy od ustaleń między stronami umowy. Jeżeli jednak strony umowy nic na ten temat nie postanowiły, standardowo twórcy należy się wynagrodzenie. Korzystanie z~utworów dostępnych na wolnych licencjach jest bezpłatne, ponieważ zapisane w~nich jest wyraźne postanowienie o~nieodpłatności („\slowoobce{royalty\dywiz{}free}”).}
} } \naglowekrozdzial{16. Wolne licencje} \akap{Szczególnym rodzajem licencji są
\wyroznienie{wolne licencje}. Wolna licencja to taka, która zezwala na nieograniczone, nieodpłatne i~niewyłączne korzystanie z~dzieł w~oryginale i~w~opracowaniu (dopuszczalne są jedynie minimalne ograniczenia tej swobody: klauzule uznania autorstwa i~klauzule copyleft, o~których piszemy w~kolejnym rozdziale). Wolne licencje to takie publiczne umowy licencyjne między twórcą i~korzystającymi utworu, które gwarantują korzystającym pełne, niczym nie ograniczone prawo do korzystania z~utworu — w~dowolny sposób, wszędzie i~zawsze.}
\akap{Autorem koncepcji wolnych licencji jest Richard Stallman. Stallman jako programista zaproponował oryginalny zestaw czterech wolności w~odniesieniu do korzystania z~programów komputerowych. Jednak zaprojektowane przez niego zasady mają charakter uniwersalny i~stosowane mogą być do budowy licencji umożliwiających korzystanie także z~innego rodzaju utworów. Wiele razy usiłowano przełożyć Cztery Wolności Stallmana na język kultury — na nasze potrzeby można to ująć na przykład tak:}
\par{} \vspace{1em} { \raggedright{}
\begin{tabularx}{\textwidth{}}{XXXXX}
1. Dostęp&Mam prawo do zapoznania się z~utworem (czyli utwór musi być jakoś rozpowszechniony)\\2. Rozpowszechnianie&Mam prawo do swobodnego rozpowszechniania kopii utworu\\3. Adaptacja&Mam prawo do tworzenia utworów zależnych na własne potrzeby\\4. Rozpowszechnianie adaptacji&Mam prawo do swobodnego rozpowszechniania utworów zależnych (czyli mam prawo do pomagania innym)\\
\end{tabularx}
} \vspace{1em} \akap{Dokumentem definiującym czym jest wolna licencja w~odniesieniu do utworów innych niż programy komputerowe jest np. Definicja Wolnych Dóbr Kultury dostępna na stronie
\href{http://freedomdefined.org}{\url{http://freedomdefined.org}}. Co do zasady tylko licencje zgodne z~tą defnicją mogą być uznawane za wolne.}
\akap{Utwory objęte wolną licencją nie muszą automatycznie być dostępne za darmo. W~języku angielskim słowo
\slowoobce{free} można rozumieć jako darmowość lub wolność od/do czegoś. Richard Stallman definiując zasady wolnego oprogramowania pisze wprost:}
\dlugicytat{ \akap{„Wolne oprogramowanie” nie oznacza „niekomercyjne”. Wolny program musi być dostępny do komercyjnego wykorzystywania, komercyjnego rozwijania i~komercyjnego rozpowszechniania. Komercyjny rozwój wolnych programów nie jest już niczym niezwykłym; takie wolne komercyjne oprogramowanie jest bardzo ważne. Możliwe, że zapłaciliście za kopie wolnych programów, mogliście też otrzymać je bezpłatnie. Ale bez względu na to, w~jaki sposób je otrzymaliście, zawsze macie wolność do kopiowania i~modyfikowania programów, a~nawet sprzedawania kopii („Definicja Wolnego Oprogramowania”).}
} \akap{Szukając informacji o~wolnych licencjach trafić możemy na takie pojęcia jak
\wyroznienie{otwarte oprogramowanie} albo \wyroznienie{otwarte zasoby}. Często w~artykułach publicystycznych łączy się cechę wolności (\slowoobce{free}) i~otwartości (\slowoobce{open}). Są to jednak dwie różne filozofie. Filzofia
\wyroznienie{otwartości} zakłada zgodę na różne, czasami daleko idące, ograniczenia praw użytkownika utworu. Filozofia stojąca za wolnymi licencjami stawia wolność na pierwszym miejscu — i~dlatego duży wysiłek włożono w~staranne definiowanie praw użytkowników. Niniejszy podręcznik nie jest najlepszym miejscem na pogłębioną analizę tego problemu, warto go jednak dla porządku zasygnalizować.}
\naglowekpodrozdzial{FAQ} \listanum{ \punkt{ \akap{ \wyroznienie{Czy wolne licencje stosować można jedynie do licencjonowania programów komputerowych?}
} \akap{Stosowanie wolnych licencji to nadawanie licencjobiorcom podstawowych praw do ich wykorzystywania, rozpowszechniania i~edytowania. Chociaż projekt Stallmana oryginalnie dotyczył dystrybucji oprogramowania, licencje zbudowane na modelu zaproponowanych przez niego wolności stosowane mogą być do licencjonowania innych utworów, także tych nieposiadających cyfrowej formy (np. drukowanych książek). Więcej na ten temat przeczytać można w~kolejnych rozdziałach.}
} \punkt{ \akap{ \wyroznienie{Po co w~ogóle stosować jakieś licencje, skoro w~internecie wszystko można znaleźć i~ściągnąć?}
} \akap{Wolności te są odbiciem kultury internetu, którego rozwój od początku opierał się na wolnej wymianie informacji. Stosowanie licencji pozwala uporządkować dystrybcję treści, dając ich użytkownikom wiedzę o~warunkach, na jakich mogą z~niej legalnie skorzystać. Bez zastosowania licencji prawnie obowiązuje ich bardzo restrykcyjne prawo autorskie, nawet jeśli powszechnie nie jest ono respektowane.}
} \punkt{ \akap{ \wyroznienie{Czy program dostępny na wolnej licencji i~oryginalnie w~internecie można legalnie rozpowszechniać na płytach CD?}
} \akap{Oczywiście. Wolne licencje dają prawo do swobodnego rozpowszechniania utworu bez względu na formę, jaką może przyjąć (plik w~internecie, wydruk, prezentacja na ekranie podczas publicznej imprezy itp.).}
} } \naglowekrozdzial{17. Ile jest wolnych licencji?} \akap{Pisaliśmy już, że tak naprawdę każdy może stworzyć swoją własną licencję i~jeśli jej zasady mieszczą się w~systemie prawa autorskiego, będzie ona legalnie obowiązywała. Katalog wolności Stallmana i~licencje GNU stały się inspiracją wielu projektów licencyjnych, często dość egzotycznych i… zabawnych. Jedną z~takich licencji jest
\slowoobce{Do What The Fuck You Want To Public License} (WTFPL), która w~praktyce przenosi licencjonowane utwory do domeny publicznej za pomocą dość zdecydowanego oświadczenia o~prawach licencjonobiorcy:}
\dlugicytat{ \akap{You just \slowoobce{DO WHAT THE FUCK YOU WANT TO}}
} \akap{Wobec rzeczywistości prawnej, w~której prawo autorskie nadaje status utworu i~obejmuje restrykcyjną ochroną nawet rysunki dwuletnich dzieci czy utwory, których autorów najprawdopodobnie nie da się już nigdy ustalić, tego typu licencja nie wygląda specjalnie absurdalnie.}
\akap{Istnieje wiele rodzajów licencji realizujących cztery wolności. Jedną z~najstarszych jest bardzo popularna, stworzona przez Richarda Stallmana oraz Ebena Moglena, licencja GNU GPL (GNU General Public License) stosowana w~dystrybucji wolnego oprogramowania. Inną taką licencją jest GFDL (GNU Free Documentation License). Jednak w~praktyce ich konstrukcja bardzo utrudniała licencjonowanie treści takich jak fotografia czy nagrania.}
\akap{Dlatego obecnie powszechnie rekomenduje się korzystanie z~dwóch najpopularniejszych wolnych licencji: CC BY (Creative Commons Uznanie Autorstwa) i~CC BY\dywiz{}SA (Creative Commons Uznanie Autorstwa\dywiz{}Na Tych Samych Warunkach). Obydwie te licencje są zgodne z~Definicją Wolnych Dóbr Kultury i~gwarantują komplet praw ich użytkownikom. O~drobnej różnicy pomiędzy nimi przeczytać można w~kolejnym rozdziale.}
\akap{Oparcie się na popularnych, zgodnych z~definicjami licencjach, ma tę zaletę, że umożliwia swobodne korzystanie z~milionów już opublikowanych na tych zasadach utworów i~— co w~wielu przypadkach jest niezmiernie ważne — swobodne łączenie ich ze sobą. Największym projektem korzystającym z~wolnej licencji CC BY\dywiz{}SA jest inicjatywa, którą z~pewnością wszyscy dobrze znają — Wikipedia, internetowa encyklopedia tworzona przez miliony ludzi na całym świecie.}
\naglowekpodrozdzial{FAQ} \listanum{ \punkt{ \akap{ \wyroznienie{Po co w~ogóle standaryzować wolne licencje?}
} \akap{Na pierwszy rzut oka projekty takie jak Definicja Wolnych Dóbr Kultury nie mają większego znaczenia dla tych, którzy po prostu chcieliby swobodnie korzystać z~zasobów Internetu. Pełnią one jednak istotną rolę, ponieważ starannie definiując niezbędne prawa użytkowników pozwalają na stworzenie mechanizmów standaryzacyjnych dla różnych typów wolnych licencji i~zapewniają ich minimalną zgodność. Dzięki temu można korzystać z~utworów opublikowanych na różnych wolnych licencjach mając za każdym razem pewność, że nasze prawa są należycie chronione.}
} \punkt{ \akap{ \wyroznienie{Czy nie byłoby prościej, gdyby istniała tylko jedna, standardowa wolna licencja?}
} \akap{Zapewne tak byłoby prościej, ale wiele projektów ma swoje specyficzne potrzeby, na które popularne licencje mogą nie odpowiadać. Zasadniczo nie zaleca się tworzenia nowych licencji, ponieważ powoduje to problemy związane z~brakiem kompatybilności prawnej. Ponieważ największym projektem wolnej kultury jest Wikipedia, dlatego wykorzystywana przez nią licencja CC BY\dywiz{}SA stała się
\slowoobce{de facto} standardem.} } } \naglowekrozdzial{18. Copyleft}
\akap{Copyleft to narzędzie pozwalające na reprodukowanie wolnej licencji w~kolejnych utworach zależnych. Mechanizm copyleftu nakłada na twórcę dzieła zależnego obowiązek udostępnienia go na tej samej licencji, na jakiej udostępniono oryginalny utwór. Może wyglądać to mniej więcej tak:}
\dlugicytat{ \akap{Fundacja korzysta z~e\dywiz{}podręcznika do matematyki dostępnego za darmo na wolnej licencji do stworzenia jego zaktualizowanej, rozszerzonej i~udoskonalonej wersji, którą następnie legalnie rozpowszechnia. Zastosowany przez twórców e\dywiz{}podręcznika mechanizm copyleftu nakłada na fundację obowiązek udostępnienia nowej wersji na tej samej wolnej licencji, co licencja oryginału.}
} \akap{Dzięki temu każdy użytkownik zaktualizowanego podręcznika ma ten sam, szeroki zakres praw, co w~przypadku podręcznika oryginalnego. Richard Stallman w~taki sposób tłumaczy ideę copyleftu:}
\dlugicytat{ \akap{Copyleft to ogólny sposób na nadanie wolności programowi lub innej pracy i~nakazanie, by wszystkie jego zmienione i~rozszerzone wersje były również wolne („Co to jest copyleft?”).}
} \akap{Copyleft nie jest konkretną licencją (w rozumieniu treści licencji, którą można by bezpośrednio wykorzystać do licencjonowania), ale ogólną zasadą wynikającą z~chęci zapewnienia jak największej wolności użytkownikom utworów zależnych. Licencja może natomiast zawierać konkretne klauzule realizujące tę zasadę, zwane „klauzulami copyleft”.}
\akap{Popularne wolne licencje CC BY i~CC BY\dywiz{}SA różnią się od siebie właśnie obecnością klauzuli copyleftowej. Licencja CC BY nie wymusza wolnego licencjonowania utworów zależnych. To oznacza, że mogą one być rozpowszechniane na dowolnych warunkach, także bardzo restrykcyjnych. Copyleftowa licencja CC BY\dywiz{}SA zobowiązuje twórców utworów zależnych, którzy je rozpowszechniają do udostępniania tych utworów na tej samej licencji.}
\naglowekpodrozdzial{FAQ} \listanum{ \punkt{ \akap{ \wyroznienie{Skąd wzięła się nazwa
\slowoobce{copyleft}?} } \akap{\slowoobce{Copyleft} to prosta gra językowa ze słowem
\slowoobce{copyright}, oznaczającym pełne zastrzeżenie praw do utworów.
\slowoobce{Copyleft} odwraca ten stan, udostępniając użytkownikowi pełną wolność korzystania z~nich. Zamiast ograniczać jego prawa, poszerza je względem standardowej formuły wszystkie prawa zastrzeżone.}
} } \naglowekrozdzial{19. Creative Commons} \akap{Creative Commons to amerykańska organizacja pozarządowa, która stworzyła własny system licencji i~wciąż zajmuje się jego rozwijaniem. Jej lokalni instytucjonalni partnerzy z~całego świata (w tym z~Polski) pracują nad dostosowywaniem postanowień licencji do specyfiki konkretnych krajowych systemów prawnych, które przecież różnią się między sobą. W~Polsce Creative Commons jest wspólnym projektem Fundacji Projekt:Polska oraz Interdyscyplinarnego Centrum Modelowania Matematycznego i~Komputerowego Uniwersytetu Warszawskiego (ICM UW).}
\akap{Licencje Creative Commons to gotowe rozwiązanie prawne, z~których można skorzystać od razu bez żadnego przygotowania prawniczego. Skorzystanie z~licencji z~punktu widzenia twórcy polega na oznaczeniu nią swojego utworu. Powoduje to udzielenie każdemu zezwolenia na korzystanie z~tego utworu w~zakresie określonym w~licencji. Z~punktu widzenia użytkownika, skorzystanie z~licencji to wykonanie określonych w~niej uprawnień w~stosunku do utworu wcześniej udostępnionego na tej licencji przez twórcę.}
\akap{Warto zauważyć, że tylko dwie z~wielu różnych licencji Creative Commons są wolne: są to CC BY oraz CC BY\dywiz{}SA. Pozostałe są licencjami dość poważnie ograniczającymi prawa użytkownika. Creative Commons dopiero od niedawna zaczęło podkreślać różnice pomiędzy licencjami wolnej kultury, a~pozostałymi. Wiele osób sądzi, że wszystkie licencje Creative Commons pozwala na w~pełni swobodne wykorzystanie utworu, ale to nie jest prawdą. Niektóre licencje Creative Commons są bardzo restrykcyjne — na przykład licencja BY\dywiz{}NC\dywiz{}ND nie pozwala na tworzenie utworów zależnych i~nie pozwala na komercyjne wykorzystanie. Jej zakres jest niewiele szerszy od tego, co i~tak mamy zagwarantowane dozwolonym użytkiem, choć posiada tę niewątpliwą zaletę, że wyraźnie potwierdza możliwość niekomercyjnego dzielenia się (co nie wynika wcale tak jasno z~przepisów o~dozwolonym użytku).}
\naglowekpodrozdzial{Rodzaje licencji Creative Commons} \akap{Wszystkie licencje systemu Creative Commons składają się z~przynajmniej jednego z~poniższych warunków, definiujących zasady korzystania z~rozpowszechnionego utworu.}
\par{} \vspace{1em} { \raggedright{}
\begin{tabularx}{\textwidth{}}{|X|X|X|X|X|}
\hline{} ikona&nazwa warunku&opis\\\hline BY&Uznanie Autorstwa&Uznanie autorstwa. Obowiązek przekazywania odbiorcom utworu określonych informacji o~twórcy (licencjodawcy), źródle utworu oraz o~samej licencji.\\\hline
SA&Na Tych Samych Warunkach&Na tych samych warunkach. Wolno rozpowszechniać utwory zależne jedynie na licencji identycznej do tej, na jakiej udostępniono utwór oryginalny.\\\hline
ND&Bez Utworów Zależnych&Bez utworów zależnych. Brak zgody na korzystanie i~rozporządzanie opracowaniami utworu.  Licencje z~tym warunkiem nie są wolnymi licencjami.\\\hline
NC&Użycie Niekomercyjne& Użycie niekomercyjne. Zakaz komercyjnego korzystania z~utworu.  Licencje z~tym warunkiem nie są wolnymi licencjami.
\\\hline
\end{tabularx}
} \vspace{1em} \akap{Organizacja Creative Commons umożliwia użycie sześciu głównych licencji:}
\par{} \vspace{1em} { \raggedright{}
\begin{tabularx}{\textwidth{}}{|X|X|X|X|X|}
\hline{} Creative Commons Uznanie autorstwa (CC BY)\\\hline Creative Commons Uznanie autorstwa\dywiz{}Na Tych Samych Warunkach (CC BY\dywiz{}SA)\\\hline
Creative Commons Uznanie autorstwa\dywiz{}Użycie niekomercyjne (CC BY\dywiz{}NC)\\\hline
Creative Commons Uznanie autorstwa\dywiz{}Bez utworów zależnych (CC BY\dywiz{}ND)\\\hline
Creative Commons Uznanie autorstwa\dywiz{}Użycie niekomercyjne\dywiz{}Na tych samych warunkach (CC BY\dywiz{}NC\dywiz{}ND)\\\hline
Creative Commons Uznanie autorstwa\dywiz{}Użycie niekomercyjne\dywiz{}Bez utworów zależnych (CC BY\dywiz{}NC\dywiz{}ND)\\\hline
\end{tabularx}
} \vspace{1em} \naglowekpodrozdzial{FAQ} \listanum{ \punkt{ \akap{
\wyroznienie{Jaka jest różnica między użyciem komercyjnym a~niekomercyjnym?}
} \akap{Ograniczenie użycia komercyjnego wynikające z~atrybutu NC w~licencjach Creative Commons sprawia wszystkim duży kłopot. Bardzo trudno jest jednoznacznie określić, na czym polega komercyjne wykorzystanie utworu. Jeśli to możliwe, lepiej unikać stosowania tych licencji Creative Commons, które zawierają ten atrybut. Być może w~kolejnych wersjach licencji Creative Commons ten atrybut zostanie usunięty.}
} \punkt{ \akap{ \wyroznienie{Udostępniłem swoje utwory na jednej z~licencji Creative Commons, ktoś je jednak wykorzystał niezgodnie z~licencją. Czy mogę zwrócić się do Creative Commons po pomoc w~rozwiązaniu tej sprawy?}
} \akap{Organizacja Creative Commons nie udziela pomocy prawnej i~wsparcia w~tego typu sytuacjach. Na pewno jednak uzyskać można podstawowe wsparcie merytoryczne w~zakresie stosowania licencji na liście dyskusyjnej. Naruszenie licencji Creative Commons jest jednak tak samo nielegalne jak naruszenie innych licencji, co oznacza, że można w~takim przypadku dochodzić określonych prawem roszczeń.}
} \punkt{ \akap{ \wyroznienie{Co się stanie, jeśli wykorzystam utwór niezgodnie z~licencją?}
} \akap{Jeśli wykorzystasz utwór niezgodnie z~licencją, np. zremiksujesz i~opublikujesz zdjęcie dostępne na licencji zakazującej korzystania z~utworów zależnych, dopuścisz się naruszenia licencji, co oznacza, że licencjodawca może skierować wobec ciebie określone prawem roszczenia. Oczywiście, w~myśl polskiego prawa możesz przygotować taki remiks, nie możesz go jednak upubliczniać.}
} \punkt{ \akap{ \wyroznienie{Dlaczego licencje Creative Commons mają różne wersje?}
} \akap{Aktualnie korzystać możemy z~licencji Creative Commons w~wersji 3.0. Creative Commons rozwija swój system licencji i~stąd kolejne wersje. Oznaczając utwór warto pamiętać nie tylko o~dodaniu informacji o~zastosowanej licencji, ale także o~jej wersji.}
} } \naglowekrozdzial{20. Licencje CC BY i~CC BY\dywiz{}SA} \akap{Dwie z~licencji z~systemu Creative Commons: Uznanie Autorstwa (CC BY) oraz Uznanie Autorstwa — Na Tych Samych Warunkach (CC BY\dywiz{}SA) są
\wyroznienie{wolnymi licencjami}.} \naglowekpodrozdzial{Uznanie Autorstwa (CC BY)}
\akap{Utwory opisane tą licencją mogą być swobodnie kopiowane, zmieniane, rozprowadzane, przedstawianie czy wykonywane, przy czym jedynym warunkiem jest poinformowanie o~twórcy (licencjodawcy) utworu, źródle oraz samej licencji. Jest to warunek w~dużej mierze zbieżny z~polską ustawą o~prawie autorskim: oznaczenie autorstwa jest jednym z~podstawowych praw każdego twórcy. Mechanizm działania licencji CC BY ilustrować może taki przykład:}
\dlugicytat{ \akap{Ania uczy się angielskiego. Znalazła w~internecie ciekawe opowiadanie fantasy. Autor tekstu udostępnił je na licencji Creative Commons: Uznanie Autorstwa (CC BY). Ania może legalnie przetłumaczyć je i~opublikować tłumaczenie na swoim blogu, informując oczywiście o~autorze oryginału, jego źródle i~licencji, na jakiej opublikowane zostało jego opowiadanie.}
} \akap{W odniesieniu do powyższego przykładu warto zwrócić uwagę na to, że Ania całkowicie legalnie mogłaby przygotować tłumaczenie nawet wtedy, gdyby autor angielskiego tekstu nie opublikował go na licencji Creative Commons — Uznanie Autorstwa. Polskie prawo pozwala na swobodne tworzenie opracowań — jednak już nie na ich publikowanie. Gdyby nie licencja CC\dywiz{}BY, Ania mogłaby przetłumaczyć opowiadanie, ale nie mogłoby ono być dostępne publicznie.}
\akap{Jak widać, wolna licencja radykalnie upraszcza korzystanie z~utworu — Ania nie musi nawet informować autora oryginalnego tekstu o~tym, że zamierza zrobić tłumaczenie i~wszystkim je udostępnić. Licencja CC\dywiz{}BY pozwala na swobodne tworzenie adaptacji, przeróbek i~remiksów oraz ich publikowanie, także w~celach komercyjnych.}
\dlugicytat{ \akap{Przetłumaczone przez Anię opowiadanie znalazło się na jej blogu i~po pewnym czasie zdobyło dużą popularność. Zauważyła je redakcja wychodzącego drukiem czasopisma o~literaturze fantastycznej i~postanowiła włączyć do kolejnego numeru.}
} \akap{Czy redakcja może tak po prostu wziąć tłumaczenie Ani i~opublikować je drukiem? To zależy na jakich zasadach Ania udostępnia swoją tłumaczenie. Licencja CC\dywiz{}BY nie wykorzystuje mechanizmu copyleftu, co oznacza, że Ania jako autorka tłumaczenia (utworu zależnego) może opublikować je na dowolnych warunkach, niekoniecznie na wolnej licencji. Jeśli tego nie zrobiła, to redakcja musi poprosić Anię o~zgodę.}
\akap{Jak widać, stosowanie wolnych licencji bez mechanizmu copyleft nie sprawia zbyt wielu kłopotów — najczęściej popełnianym błędem jest tu nieprawidłowe oznaczenie autorstwa. Ponieważ problem oznaczania autorstwa i~licencji dotyczy wszystkich wolnych licencji, o~tym, w~jaki sposób robić to prawidłowo piszemy w~jednym z~kolejnych modułów.}
\naglowekpodrozdzial{Uznanie Autorstwa — Na Tych Samych Warunkach (CC BY\dywiz{}SA)}
\akap{Licencja Creative Commons Uznanie autorstwa\dywiz{}Na tych samych warunkach (CC BY\dywiz{}SA) zawiera w~sobie mechanizm copyleft. Oznacza to, że twórcy rozpowszechniający swoje utwory zależny zbudowane na bazie oryginałów dostępnych na CC BY\dywiz{}SA muszą licencjonować je na tej samej licencji. Jeśli odniesiemy to do przykładu Ani i~jej tłumaczenia okaże się, że w~przypadku zastosowania licencji copyleftowej również nie będzie problemów z~dalszym wykorzystywaniem utworu:}
\dlugicytat{ \akap{Ania uczy się angielskiego. Znalazła w~internecie bardzo ciekawe opowiadanie, które autor udostępnił na licencji Creative Commons: Uznanie Autorstwa\dywiz{}Na tych samych warunkach (CC BY\dywiz{}SA). Ania może legalnie przetłumaczyć je i~opublikować tłumaczenie na swoim blogu, informując oczywiście o~autorze oryginału oraz stosując do tłumaczenia tę samą co on licencję — CC BY\dywiz{}SA.}
} \akap{Oryginalna, anglojęzyczna wersja opowiadania dostępna jest na licencji Uznanie Autorstwa\dywiz{}Na tych samych warunkach. Tłumaczenie Ani — zgodnie z~tą licencją — musi być upublicznione na tych samych zasadach:}
\dlugicytat{ \akap{Przetłumaczone przez Anię opowiadanie znalazło się na jej blogu i~po pewnym czasie zdobyło dużą popularność. Zauważyła je redakcja wychodzącego drukiem czasopisma o~literaturze fantastycznej i~postanowiła włączyć do kolejnego numeru. Ponieważ opowiadanie opublikowane jest na licencji CC BY\dywiz{}SA redakcja nie musi już uzyskać dodatkowej zgody, tylko po prostu włącza opowiadanie do numeru. Przy czym, redakcja musi podać informację o~autorstwie oraz źródle, a~także zachować informację o~licencji CC\dywiz{}BY\dywiz{}SA, przez co tłumaczenie nie może zostać przez nich zawłaszczone.}
} \naglowekpodrozdzial{FAQ} \listanum{ \punkt{ \akap{ \wyroznienie{Czy licencjonując swój utwór licencją CC BY lub CC BY\dywiz{}SA muszę podawać swoje pełne imię i~nazwisko?}
} \akap{Autor może podpisać się pseudonimem. Teoretycznie wcale się nie podpisywać, ale to może komplikować korzystanie z~tak rozpowszechnionego utworu.}
} \punkt{ \akap{ \wyroznienie{Którą wolną licencję lepiej stosować — CC BY czy CC BY\dywiz{}SA?}
} \akap{Licencja CC BY\dywiz{}SA lepiej chroni prawa użytkowników utworów zależnych. Jeśli zależy nam na umożliwieniu monopolistycznego wykorzystania utworów zależnych, to lepsza będzie licencja CC BY. Jeśli chcemy uniemożliwić zawłaszczanie utworów zależnych, to lepsza będzie licencja CC BY\dywiz{}SA.}
} } \naglowekrozdzial{21. Dobre i~złe praktyki opisu wolnych licencji}
\akap{Jedną z~podstawowych umiejętności związanych z~posługiwaniem się wolnymi licencjami jest poprawne opisywanie za ich pomocą licencjonowanych utworów. Błędny opis może utrudnić lub uniemożliwić korzystanie z~utworu zgodnie z~licencją. Jakie wobec tego są zasady poprawnego opisu licencji udostępnianego utworu? Informacje te ważne są nie tylko dla twórców, ale także osób, które chcą korzystać z~rozpowszechnianych na wolnych licencjach treści.}
\naglowekpodrozdzial{Elementy opisu licencji} \akap{Warto podkreślić, że nie istnieje jeden powszechnie obowiązujący, ortodoksyjny schemat oznaczania utworów na wolnych licencjach. W~dystrybucji cyfrowej oraz w~utworach rozpowszechnianych na fizycznych nośnikach stosuje się rozmaite modele opisu licencji, zazwyczaj racjonalnie dostosowane do charakteru utworu i~właściwości jego nośnika. Pomimo tego można bez problemu wymienić kilka elementów, które powinny znaleźć się w~informacji licencyjnej bez względu na to, czy korzystamy z~udostępnianego na wolnej licencji utworu czy też taki utwór tworzymy i~publikujemy:}
\par{} \vspace{1em} { \raggedright{}
\begin{tabularx}{\textwidth{}}{|X|X|X|X|X|}
\hline{} Element opisu licencji\\\hline Informacja o~autorze lub autorach, albo o~innym niż autor(rzy) licencjodawcy\\\hline
Źródło (adres bibliograficzny z~tytułem utworu lub link, jeśli wykorzystywany utwór jest dostępny w~sieci)\\\hline
Dokładna nazwa licencji z~odnośnikiem do strony licencji w~serwisie Creative Commons lub — w~przypadku druku — adresem URL do niej\\\hline
\end{tabularx}
} \vspace{1em} \srodtytul{1. Informacja o~autorze} \akap{Zgodnie z~polskim prawem zawsze zobowiązani jesteśmy do odpowiedniego informowania o~twórcy wykorzystywanego utworu. Dotyczy to także licencjonowania utworów zależnych, powstałych na bazie oryginałów stworzonych przez kogoś innego. Oczywiście twórca lub osoba uprawniona może zażyczyć sobie, aby wymieniano go używając jego pseudonimu a~nie imienia i~nazwiska — lub aby nie informowano o~nim wcale.}
\akap{W przypadku utworu posiadającego wielu autorów wymieniamy ich wszystkich, chyba, że radykalnie zwiększy to objętość informacji licencyjnej. Wówczas zastosować można znane z~opisów bibliograficznych skróty (np. i~in.) lub udostępnić link do strony z~informacją o~wszystkich autorach, jak to robi się w~przypadku wykorzystywania haseł z~Wikipedii.}
\srodtytul{2. Informacja o~źródle} \akap{Publikując utwór zobowiązani jesteśmy także do podania informacji pozwalającej na zidentyfikowanie jego źródła. Tradycyjnie taką informację podaje się w~formie adresu bibliograficznego — podając tytuł utworu, a~czasami także wydawcę, miejsce i~rok wydania. W~czasach internetu adres bibliograficzny coraz częściej zastępowany jest po prostu przez link do źródła, co pozwala precyzyjnie określić pochodzenie materiału.}
\akap{Licencjodawca może wymagać, aby przy opisie licencji pojawiła się informacja np. o~instytucji, która odpowiedzialna jest za publikację utworu. Zakresem klauzul BY w~licencjach CC objęte mogą być różne inne instytucje wskazane przez Licencjodawcę. To samo dotyczy obowiązkowego zamieszczania logo.}
\srodtytul{3. Informacja o~licencji} \akap{Precyzyjne określenie licencji, na której dostępny jest utwór, jest niezmiernie ważne, bo dzięki temu użytkownicy znają szczegółowo reguły, których powinni przestrzegać. Powinna więc to być precyzyjna nazwa licencji, np. „Licencja Wolnej Dokumentacji GNU v. 3” albo „Creative Commons Uznanie Autorstwa — Na Tych Samych Warunkach 3.0 PL”. Dopuszczalne jest użycie skrótu, np. GFDL lub CC BY\dywiz{}SA. Konieczne jest także podanie linku do pełnego tekstu licencji. Odnośnik taki musi być bezwzględnie umieszczony nawet przy utworach rozpowszechnianych na fizycznych nośnikach (np. w~publikacjach drukowanych).}
\akap{Niektóre wolne licencje, takie jak GFDL, wymagają także załączenia pełnego tekstu licencji razem z~utworem.}
\akap{Przy oznaczaniu licencji wiele osób popełnia błędy. Na przykład:}
\dlugicytat{ \akap{Wszystkie wpisy na blogu dostępne są na licencji Creative Commons.}
} \akap{Creative Commons udostępnia cały katalog różnych licencji — powyższy wpis nie informuje, o~jaką konkretnie licencję chodzi, z~więc… jest zupełnie bezużyteczny.}
\akap{W razie potrzeby w~opisie licencji umieścić można także jedną z~ikon symbolizujących licencję. Wykorzystanie znaku graficznego nie powinno być jednak ważniejsze od poprawnego zapisu pełnej informacji o~licencji.}
\akap{Jeśli licencjonowany utwór dostępny jest w~internecie na stronie WWW, w~opisie licencji zastosować można specjalny kod dla wyszukiwarek. W~przypadku licencji Creative Commons można go wygenerować w~formularzu wyboru licencji, dostępnym na stronie creativecommons.org. Umieszczony na stronie kod informować będzie roboty indeksujące internet, że dany materiał dostępny jest na konkretnych warunkach licencyjnych. Ułatwia to budowanie wyszukiwarek tego typu zasobów.}
\naglowekpodrozdzial{Przykłady poprawnych informacji licencyjnych}
\srodtytul{1. Hasło z~Wikipedii} \akap{Załóżmy, że właściciel niewielkiego pensjonatu w~Gdańsku chce na stronie internetowej umieścić artykuł o~historii Gdańska, aby zachęcić turystów do przyjazdu i~oferowanych przez niego noclegów. Wykorzystuje dostępne w~Wikipedii hasło „Historia Gdańska”. Aby zrobić to legalnie, dodaje na stronie czytelną informację o~warunkach, na jakich skorzystał z~tej encyklopedii:}
\dlugicytat{ \akap{Źródło: Wikipedia, \href{http://pl.wikipedia.org/wiki/Historia\%20Gda\%C5\%84ska}{\url{http://pl.wikipedia.org/wiki/Historia Gdańska}}, autorzy:
\href{http://pl.wikipedia.org/w/index.php\%3Ftitle\%3DHistoria\_Gda\%25C5\%2584ska\%26action\%3Dhistory}{\url{http://pl.wikipedia.org/w/index.php?title=Historia\_Gda\%C5\%84ska\&action=history}}
licencja: CC-BY-SA 3.0, \href{http://creativecommons.org/licenses/by-sa}{\url{http://creativecommons.org/licenses/by-sa}}}
} \akap{Dlaczego zastosowano tutaj link do strony z~historią edycji hasła? To sposób na wymienienie wszystkich autorów. Zamiast linku można by oczywiście wypisać ich wszystkich, nie wyglądałoby to jednak najlepiej:}
\dlugicytat{ \akap{\textcopyright{} Źródło: Wikipedia, \href{http://pl.wikipedia.org/wiki/Historia\%20Gda\%C5\%84ska}{\url{http://pl.wikipedia.org/wiki/Historia Gdańska}}, autorzy: Artur Andrzej, KamikazeBot, Mudzo, Pawski, Airwolf, Oruniak, Belissarius, Szczureq, Beno, MZM, Mmkay, Mzopw i~in., licencja: CC-BY-SA 3.0,
\href{http://creativecommons.org/licenses/by-sa}{\url{http://creativecommons.org/licenses/by-sa}}}
} \akap{Wikipedystyczne hasło 'Historia Gdańska' posiada setki autorów. Co więcej, na stronie prezentującej rejestr zmian podpisani są nie tylko edytorzy, ale też boty — skrypty automatycznie poprawiające drobne błędy np. w~strukturze hasła. Na pierwszy rzut oka bardzo trudno odróżnić je od zwykłych wikipedystów. Wygodniej jest po prostu udostępnić link do pełnej listy autorów.}
\srodtytul{2. Fotografia z~prywatnego bloga wykorzystana na portalu}
\akap{Duże portale internetowe wykorzystują dziś wiele materiałów pochodzących bezpośrednio od internautów — bardzo często robią to błędne i~wykazują się przy tym dużą nieuczciwością. Jeśli już w~ogóle informują o~źródle wykorzystanego przez siebie materiału (np. fotografii z~jakiegoś wydarzenia, które zostało zrobione przez przypadkową osobę), piszą:}
\dlugicytat{ \akap{Źródło: Internet} } \akap{Odwróćmy tę sytuację: w~niniejszym opracowaniu wykorzystać chcielibyśmy jakiś materiał np. z~gazety „Wieści codzienne”, poświęcony aktualnym wyzwaniom związanym z~prawem autorskim. Gdybyśmy działali tak jak niektórzy dziennikarze z~popularnych portali, przygotowana przez nas informacja wyglądałaby tak:}
\dlugicytat{ \akap{Źródło: prasa} } \akap{Robiąc to oczywiście złamalibyśmy zasady prawa autorskiego — bez zgody autora czy wydawcy nie mielibyśmy nawet prawa republikować tekstu, nie mówiąc już o~poprawnym opisie warunków jego rozpowszechniania. Fotografia opublikowana na blogu i~profesjonalny tekst dziennikarski pod względem prawa autorskiego formalnie niczym się nie różnią — to wciąż utwory, które podlegają określonej ochronie.}
\akap{Wróćmy do zdjęcia i~postarajmy się jej poprawnie opisać. Pojawia się tutaj dość istotny problem — nigdzie nie możemy znaleźć informacji o~imieniu i~nazwisku autora bloga — znamy tylko jego pseudonim. W~stopce bloga znajduje się ikona informująca o~licencji CC BY\dywiz{}SA. Czy jednak licencja ta dotyczy całej treści bloga?}
\akap{Ten przykład pokazuje, jak ważne jest poprawne opisanie licencji także przez licencjodawcę. Umieszczenie na stronie ikony z~informacją o~licencji nie wystarcza — nie wiemy, czego dokładnie ona dotyczy. Starając się jak najbardziej poprawnie przygotować informację o~warunkach, na jakich chcemy skorzystać ze zdjęcia, powinniśmy spróbować skontaktować się z~autorem i~wyjaśnić wszystkie wątpliwości. A~przecież wartość wolnych licencji polega także na tym, że powinniśmy mieć możliwość skorzystania z~utworu bez żadnych formalności. Źle przygotowany opis blokuje taką możliwość.}
\akap{W naszym przykładzie, po krótkiej wymianie maili z~autorem uzyskujemy potwierdzenie tego, że fotografia dostępna jest na licencji Creative Commons Uznanie autorstwa 3.0 Polska. Autor nie podaje jednak swojego imienia i~nazwiska — ma do tego prawo. Ostatecznie przygotowujemy następujący opis, który umieszczamy przy opublikowanym na portalu zdjęciem:}
\dlugicytat{ \akap{Pociąg na stacji Warszawa Śródmieście WKD, fot. bart32, źródło
\href{http://bart32.blox.pl/\%3Fp\%3D732}{\url{http://bart32.blox.pl/?p=732}}, licencja: Creative Commons Uznanie autorstwa 3.0 Polska (\href{http://creativecommons.org/licenses/by/3.0/pl/}{\url{http://creativecommons.org/licenses/by/3.0/pl/}}).}
} \akap{W opisie znalazła się informacja o~autorze (pseudonim), źródle oraz wykorzystanej licencji (wraz z~obowiązkowym odnośnikiem do tekstu licencji). Gdybyśmy na bazie tej fotografii chcieli przygotować jakiś utwór zależny — choćby infografikę opisującą liczbę pasażerów korzystających z~usług Warszawskiej Kolei Dojazdowej (WKD), jej opis uwzględniać powinien informację o~wykorzystywanym zdjęciu:}
\dlugicytat{ \akap{Liczba pasażerów WKD w~latach 2009–2011. Wykorzystano zdjęcie 'Pociąg na stacji Warszawa Śródmieście WKD', fot. bart32, źródło
\href{http://bart32.blox.pl/\%3Fp\%3D732}{\url{http://bart32.blox.pl/?p=732}}, licencja: Creative Commons Uznanie autorstwa 3.0 Polska (\href{http://creativecommons.org/licenses/by/3.0/pl/}{\url{http://creativecommons.org/licenses/by/3.0/pl/}}).}
} \srodtytul{3. Remiks} \akap{Pozostańmy jeszcze przy problemie opisu licencji utworów zależnych. Co zrobić, jeśli tych utworów jest naprawdę dużo — tak jak w~przypadku remiksu muzycznego?}
\dlugicytat{ \akap{Do stworzenia jednego z~nagrań na swoją debiutancką płytę zespół wykorzystał fragmenty kilkunastu piosenek dostępnych w~serwisie jamendo.com na wolnych licencjach CC BY i~CC BY\dywiz{}SA.}
} \akap{Jak w~takim przypadku zachować porządek i~odpowiednio informować o~wykorzystanych materiałach? W~przygotowanej przez australijski oddział Creative Commons broszurze znaleźć można prosty pomysł na ułatwienie sobie pracy z~wieloma utworami i~wieloma licencjami. Wystarczy przygotować prostą tabelkę:}
\par{} \vspace{1em} { \raggedright{}
\begin{tabularx}{\textwidth{}}{|X|X|X|X|X|}
\hline{} Autor&Tytuł&Źródło&Licencja\\\hline Josh Woodward&Fit For a~King&\href{http://www.jamendo.com/pl/track/301969/fit-for-a-king}{\url{http://www.jamendo.com/pl/track/301969/fit-for-a-king}}&CC BY 3.0\\\hline
Frozen Silence&Midwives&\href{http://www.jamendo.com/pl/track/25225/midwives}{\url{http://www.jamendo.com/pl/track/25225/midwives}}&CC BY 3.0\\\hline
Song to Paris&The Dada Weatherman&\href{http://www.jamendo.com/pl/track/353349/song-to-paric}{\url{http://www.jamendo.com/pl/track/353349/song-to-paric}}&CCBY\dywiz{}SA 3.0\\\hline
\end{tabularx}
} \vspace{1em} \akap{Przygotowanie takiego spisu zdecydowanie ułatwi sporządzenie odpowiedniego opisu licencyjnego, który zamieścić można np. na wewnętrznej okładce płyty czy na stronie, na której udostępniany jest utwór. Warto zwrócić uwagę, że wykorzystano tu piosenkę dostępną na licencji Uznanie autorstwa\dywiz{}Na tych samych warunkach. Oznacza to, że remiks również musi być dostępny na tej licencji.}
\naglowekpodrozdzial{Jak opisać licencjonowany utwór w~różnych mediach?}
\par{} \vspace{1em} { \raggedright{}
\begin{tabularx}{\textwidth{}}{|X|X|X|X|X|}
\hline{} Typ nośnika&Forma opisu licencji\\\hline Książki, czasopisma i~magazyny&Dołącz odpowiednią informację o~licencji obok wykorzystywanego utworu (zdjęcia, tekstu) lub w~stopce na stronie, na której został on wykorzystany. Ewentualnie istnieje możliwość wyliczenia wszystkich wykorzystanych utworów wraz z~zasadami, na jakich zostały udostępnione, na końcowych stronach wydawnictwa.\\\hline
Fotografie i~grafiki&Umieść odpowiednią informację na lub tuż przy wykorzystywanej grafice lub zdjęciu. Informacja ta znaleźć się też może w~stopce strony, jeśli inne formy atrybucji będą niepotrzebnie ingerować w~treść materiału graficznego.\\\hline
Prezentacja (slajdy)&Umieść informacje o~wykorzystywanym materiale bezpośrednio na slajdzie, na którym jest on republikowany — lub na ostatniej stronie całej prezentacji.\\\hline
Film&Umieść informacje o~wykorzystywanym materiale bezpośrednio w~treści filmu w~czasie, w~którym jest on wyświetlany — lub w~napisach końcowych z~zaznaczeniem, w~jakiej części filmu dany fragment jest republikowany.\\\hline
Podcast, nagranie audio&Poinformuj słuchaczy o~wykorzystywaniu materiałów objętych wolną licencją w~trakcie audycji, na jej początku, końcu lub tuż przed emisją republikowanego materiału dźwiękowego.\\\hline
\end{tabularx}
} \vspace{1em} \akap{[Opracowanie na podstawie \href{http://creativecommons.org.au/content/attributingccmaterials.pdf}{\url{http://creativecommons.org.au/content/attributingccmaterials.pdf}}, CC BY The Australian Research Council Centre of Excellence for Creative Industries and Innovation, Creative Commons Australia,
\href{http://creativecommons.org/licenses/by/2.5/au/}{\url{http://creativecommons.org/licenses/by/2.5/au/}}]}
\naglowekrozdzial{22. Wzory umów licencyjnych} \akap{Poprawny opis licencji warunkuje korzystanie z~utworów rozpowszechnianych na wolnych licencjach. Instytucje czy przedsiębiorstwa, które chciałyby wdrożyć wolne licencje do podpisywanych przez siebie umów prawnoautorskich również powinny je odpowiednio poprawnie przygotowywać. Nieodpłatne korzystanie z~wolnych licencji (a więc także licencjonowanie utworu) nie wymaga podpisywania żadnych dokumentów, ale w~przypadku, kiedy płacimy komuś za wykonanie utworu,który ma być dostępny na wolnej licencji, zdecydowanie warto podpisać umowę o~dzieło prawno\dywiz{}autorskie (chociażby z~powodu rozliczeń podatkowych). Wówczas w~umowie o~przeniesienie praw autorskich czy w~umowie licencyjnej pojawić się muszą zapisy związane z~zasadami wybranej przez nas wolnej licencji.}
\akap{Wzory takich umów publikuje na swojej stronie Koalicja Otwartej Edukacji, porozumienie organizacji pozarządowych i~instytucji publicznych działających w~obszarze edukacji, nauki i~kultury.}
\akap{Na stronie \href{http://koed.org.pl/wzory-umow/}{\url{http://koed.org.pl/wzory-umow/}}
znaleźć można zestaw szablonów umów prawnoautorskich, które przewidują udostępnianie utworów na wybranych licencjach Creative Commons. Przygotowano wersje dla umów odpłatnych i~nieodpłatnych. Szablony te wykorzystywać można w~pracy własnej organizacji czy instytucji za darmo i~bez konieczności jakiejkolwiek rejestracji.}
} \def\coverby{
                %
                }
                \def\editors{\par{Redakcja techniczna:
    Aneta Rawska}} \editorialsection{}
\end{document}
